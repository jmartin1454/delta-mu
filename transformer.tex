
\subsection{Transformer Core Measurements}
In second approach, the witness cylinder was used as the core of a
transformer. Two coils (primary and sense coils) were wound on the
witness cylinder. The primary coil generated an AC magnetic field
using the internal oscillator of an SRS830 lock-in amplifier while the
sense coil measured the sense voltage and was picking up the magnetic
field flux produced by the primary coil. This experiment was done at 1
Hz and as small as possible $H$,
% state typical value
to measure the slope of the minor $B-H$ loops near the origin of the
$B-H$ space. In lower frequencies, noise dominated the actual signal.
% why?
The temperature of the core was measured concurrently as the sense
voltage using non-magnetic type T thermocouples.  The measured voltage
induced on the sense coil, is proportional to $\mu$.  The measurements
show a range of 0.1\%/K to 2\%/K for $\frac{1}{\mu}\frac{d\mu}{dT}$,
again assuming the material to be linear.

In all transformer measurements, the sense voltage from the sense coil
was demodulated by a lock-in amplifier. In
Fig.~\ref{fig:data_and_simulation} the in-phase component $Y$ and
out-of-phase component $X$ of the sense signal as a function of
applied $H$ field at 1 Hz are shown. In the ideal case where the
material is linear, the $X$ component should be zero. The ratio of the
$X/Y$ depends on both the applied frequency and the position on the
$B-H$ region. We were interested in regions where $\vert Y \vert >
\vert X \vert $ in which the material behaves as being linear and the
$\vert H \vert $ values correspond to $\vert X \vert < \vert X_{max}
\vert$ while still being able to get a good signal to noise ratio.


A concern in both measurements is that, the material is not
linear. There are other effects such as hysteresis, saturation, eddy
currents and skin depth which all contribute to the measured signal
and so it is not possible to embody the results as a single parameter.
To check the sensitivity of the sense coil demodulated voltage to
these effects, a theoretical model of the hysteresis (Jiles-Atherton
model) was used \cite{bib:jiles}, see Fig.
\ref{fig:data_and_simulation}.

The key parameters of Jiles-Atherton model are saturation
magnetization $M_s$, mean field parameter $\alpha$ which represents
interdomain coupling, $a$ with the dimensions of magnetic field which
characterizes the shape of anhysteretic magnetization, bulk
magnetization $M$ and the magnetic field $H$. A shown in
Fig. \ref{fig:data_and_simulation}, trends in the measurements and
simulations are fairly consistent.

Some measurements regarding degaussing have been done to study how
removing magnetization effects the sense voltage. Degaussing
(idealizing) magnetic shields means to saturate the shields by
applying a very large magnetic field and then slowly ramping this
applied field down to zero.
% State how done. 
In most cases, degaussing helped to
increase the signal and hence $\mu$.
% But it didn't change its temperature dependence significantly.  Poor
% degaussing could result in initial slopes that were large for a
% while?

\begin{figure}[h!]
\begin{center}
   \includegraphics[width=0.5\textwidth]{data_and_simulation3.PNG}
    \caption{These graphs represent the sense voltage as a function of applied $H$ field at 1 Hz. Graph (a) shows the out-of-phase component $X$ of the measured sense voltage as well as the simulation. Graph (b) shows the in-phase component $Y$ of the signal and its simulation.}
    \label{fig:data_and_simulation}
    \end{center}
\end{figure} 

\subsubsection{Systematic Errors}

% Need list
% Please place here
%
% 1.  Focus on unique ones:
% - motion of wires
% - degaussing, how done, and general observations
% - resistance of wires
% - different numbers of turns on primary/secondary
% - in-phase/out-of-phase components, how small should in-phase be?
% - ...

% 2.  Or if there are any other ones that are from the same source
% (e.g. resistance of wire might be in this category) but the value is
% different for some reason

% 3.  Or if some are eliminated, it would be good to note here, for
% example, fluxgate systematics and some classes of motion systematics

The dominant source of systematic uncertainty in this method arises
from the properties of witness cylinders. Saturation and magnetization
of the witness cylinders and the nonlinearity of $/mu$ in general had
a considerable effect on the measured signal. Despite of the
consistent general trend of $B(T)$, different witness cylinders tend
to behave differently under similar conditions. The applied voltage on
the primary coil was stable and was measured across a 1~$\Omega$
temperature controlled resistor. To reduce the background noise and
thermally isolate the witness cylinder, some measurements were done by
placing the witness cylinder inside the big passive shields. The
results did not show a significant change.

%\begin{itemize}
%\item Describe experimental setup and important consdierations (e.g. relationship of data to effective $%\mu$)
%\item Explain B, H, f, and dominant systematic effects.
%\item Understanding in terms of Jiles-Atherton.  Complications that
 % this does not translate well into ``$\mu$''.
%\item One data graph?  (Perhaps the Jiles-Atherton one?)
%\item State overall result and systematic error.
%\end{itemize}
%%%%%%%%%%%%%%%%%%%%%%%%%%%%%%%%%%%%%%%%%%%%%%%%%%%%%%%%%%%%%%%%%%%%%%
% Taraneh, can you write text up to this point?
%%%%%%%%%%%%%%%%%%%%%%%%%%%%%%%%%%%%%%%%%%%%%%%%%%%%%%%%%%%%%%%%%%%%%%
