\documentclass[review]{elsarticle}
\usepackage{lineno,hyperref,amssymb,graphicx,booktabs}

\modulolinenumbers[5]
\journal{Nuclear Instruments and Methods in Physics Research A}


% Note from Jeff: Our previous NMOR article was 20 pages in this
% format, which became about 6 pages in the final paper in NIMA.


%%%%%%%%%%%%%%%%%%%%%%%
%% Elsevier bibliography styles
%%%%%%%%%%%%%%%%%%%%%%%
%% To change the style, put a % in front of the second line of the current style and
%% remove the % from the second line of the style you would like to use.
%%%%%%%%%%%%%%%%%%%%%%%

%% Numbered
%\bibliographystyle{model1-num-names}

%% Numbered without titles
%\bibliographystyle{model1a-num-names}

%% Harvard
%\bibliographystyle{model2-names.bst}\biboptions{authoryear}

%% Vancouver numbered
%\usepackage{numcompress}\bibliographystyle{model3-num-names}

%% Vancouver name/year
%\usepackage{numcompress}\bibliographystyle{model4-names}\biboptions{authoryear}

%% APA style
%\bibliographystyle{model5-names}\biboptions{authoryear}

%% AMA style
%\usepackage{numcompress}\bibliographystyle{model6-num-names}

%% `Elsevier LaTeX' style
%\bibliographystyle{elsarticle-num}
%%%%%%%%%%%%%%%%%%%%%%%

\begin{document}

\begin{frontmatter}

\title{Sensitivity of Fields within Magnetically Shielded Volumes to
  Changes in Permeability}

\author[manitoba]{T. Andalib\corref{mycorrespondingauthor}}
\cortext[mycorrespondingauthor]{Corresponding author}
\ead{andalibt@myumanitoba.ca}
\author[winnipeg,manitoba]{C.P. Bidinosti}
\author[winnipeg,manitoba]{R.R. Mammei}
\author[winnipeg,manitoba]{J.W. Martin}
\author[winnipeg]{D. Ostapchuk}

\address[winnipeg]{Physics Department, The University of Winnipeg, 515 Portage Avenue, Winnipeg, MB, R3B 2E9, Canada}
\address[manitoba]{Department of Physics and Astronomy, University of Manitoba, Winnipeg, MB R3T 2N2, Canada}


\begin{abstract}
Future experiments seeking to measure the neutron electric dipole
moment (nEDM) require stable and homogeneous magnetic fields.  The
stability of the magnetic field within a magnetically shielded volume
is influenced by a number of factors.  In this paper, we study one of
these factors, which is the dependence of the internally generated
field on the permeability of the material.  We also provide
measurements of the temperature-dependence of the permeability of the
material, and indicate the extrapolation yet required to adequately
use these measurements to design future nEDM experiments.
\end{abstract}

\begin{keyword}
Magnetic Shielding \sep Neutron Electric Dipole Moment \sep Magnetic Field Stability
\end{keyword}

\end{frontmatter}

\linenumbers

\section{Introduction}

The next generation of neutron electric dipole moment (EDM)
experiments aim to measure the EDM $d_n$ with proposed precision
$\delta d_n\lesssim
10^{-27}$~e-cm~\cite{bib:nedm1,bib:nedm2,bib:nedm2.5,bib:nedm3,bib:nedm3.5,bib:nedm4,bib:nedm5,bib:nedm6,bib:nedm6.5}.
In the previous best experiment \cite{bib:pendlebury}, which discovered
$d_n<3\times 10^{-26}$~e-cm (90\% C.L), effects related to magnetic field
homogeneity and instability were found to dominate the systematic
error.  A detailed understanding of passive and active magnetic
shielding, magnetic field generation within shielded volumes, and
precision magnetometry is expected to be crucial to achieve the
systematic error goals for the next generation of experiments.  Much
of the R\&D effort for these experiments is focused on careful design
and testing of various magnetic shield geometries with precision
magnetometers~\cite{bib:brys,bib:afach,bib:fierlingerroom,bib:sturmthesis,bib:patton}.

%\begin{itemize}
%\item General requirements on field, stability of field, stability of gradient.
%\item List all possible factors that can affect field stability -
 % degaussing, vibrations, etc. possibly with additional references.
%\item State the problem addressed in this paper (changes in $\mu$) and
 % the result.
%\end{itemize}
The nEDM experiment at TRIUMF aims to determine $\delta d_n \sim
10^{-27}$ e$\cdot$cm. This level of precision requires a 1 $\mu$T
magnetic field with the homogeneity of $<$ 1 nT/m and the stability of
pT over the UCN free-precession time.

%%%%%%%%%%%%%%%%%%%%%%%%%%%%%%%%%%%%%%%%%%%%%%%%%%%%%%%%%%%%%%%%%%%%
% Taraneh
%%%%%%%%%%%%%%%%%%%%%%%%%%%%%%%%%%%%%%%%%%%%%%%%%%%%%%%%%%%%%%%%%%%%
% 
% general layout of magnetic system for an EDM experiment
%

The magnetic field system for the future nEDM experiment at TRIUMF comprises the components used to null environment magnetic fields and provide a stable homogeneous static magnetic field. It includes an active compensation system, a magnetically shielded room (MSR), several more layers of passive shielding, and  an internal coil system.
The active compensation system will be used to reduce the external magnetic field at the measurement site to $\mu$T level and prevent the saturation of the passive magnetic shields.
The passive shielding system nullifies the residual background fields to the pT level with a two-stage system including a 2-layer MSR and a smaller 3-layer shield that fits inside the room and surrounds the nEDM apparatus.
Mechanical and temperature changes of this system and the demagnetization procedure of the passive magnetic shielding will affect the stability of the magnetic field within the measurement volume\cite{bib:thiel,bib:altarev2014,bib:altarev2015,bib:voigt}.

%
% demagnetization:  cite Thiel et al., recent Fierlinger paper
%

%
% mech/temp:  cite BMSR paper
%

%
% mech - other citations?  What have other people done to characterize
% their MSR's?
%


The passive shields for the nEDM experiments are made of highly
permeable materials such as Mu-Metal which is an Ni-Fe alloy that
allows measurements at low frequency magnetic fields. These alloys
operate at room temperature and are sensitive to small temperature
variations which affects the electromagnetic behaviour of the material
such as the magnetic permeability, $\mu$ \cite{bib:couderchon,bib:kruppvdm}. For
these type of alloys $\mu$ depends also on the annealing process
\cite{bib:gupta,bib:bozorth}. The temperature dependence of $\mu$ has been
measured from two different approaches and the overall result found to
be 0.1\%/K $<\frac{1}{\mu} \frac{d\mu}{dT}<$2.3\%/K.

%%%%%%%%%%%%%%%%%%%%%%%%%%%%%%%%%%%%%%%%%%%%%%%%%%%%%%%%%%%%%%%%%%%%
% Taraneh
%%%%%%%%%%%%%%%%%%%%%%%%%%%%%%%%%%%%%%%%%%%%%%%%%%%%%%%%%%%%%%%%%%%%
% 
% more literature?
%


\section{Sensitivity of Internally Generated Field to Permeability of the Shield $B_0(\mu)$}

\subsection{Analytical Calculations in Spherical (and Cylindrical) Geometry}

%%%%%%%%%%%%%%%%%%%%%%%%%%%%%%%%%%%%%%%%%%%%%%%%%%%%%%%%%%%%%%%%%%%%55
% Taraneh
%%%%%%%%%%%%%%%%%%%%%%%%%%%%%%%%%%%%%%%%%%%%%%%%%%%%%%%%%%%%%%%%%%%%55
%\begin{itemize}
%\item State basic physics of problem, define reaction factor, note
%  return flux through shield, etc.
%\item State formulae and results for reasonable parameters.
%\item Possibly one graph of $B_0(\mu)$ with both geometries.
%\end{itemize}

The presence of a coil inside the innermost passive shield turns the shield into a return yoke and it is due to the penetration of the magnetic field flux into the magnetic shield. The ratio of the magnetic field inside the coil in the presence of the magnetic shield to the coil in free space can be called the reaction factor $C$ which is $>$ 1 for spherical and cylindrical geometries~\cite{bib:bidinosti}. The coupling of the coil to the innermost layer of the magnetic shield suggests that any changes in the properties of the magnetic shield gives rise to a change in the internal field. One of these properties is $\mu$ which is of significant importance because of the high permeability of the passive shields. As a result, the stability of the magnetic shields contributes to the stability of the internal field.

%From Refs. \cite{bib:smythe, bib:ferraro} for a zonal surface current
%\begin{equation}
%\bold{F}=- \sum_{n=1} ^{\infty} \frac{C_n}{\ell} P^1 _n (u) \hat{\phi}
%\end{equation}
%bound on a sphere with radius $\ell$ the magnetic field at $r < \ell$ and $r>\ell$ is calculated.
%The case of an ideal spherical surface current ($n=1$) inside a spherical shell with inner radius $a$ and outer radius $b$ has been analytically calculated considering the following boundary conditions

%\begin{equation}
%\frac{1}{\mu_0} B_{\theta} (a^-) = \frac{1}{\mu} B_{\theta}(a^+)
%\end{equation}
%\begin{equation}
%\frac{1}{\mu}B_{\theta}(b^-)=\frac{1}{\mu_0}B_{\theta}(b^+).
%\end{equation}
%Fig. \ref{fig:Magnetic_Field} shows the magnetic field $B$ as a function of relative $\mu$ for $\ell=0.53$ m, $a=0.57$ m and a thickness of 1.5 mm for the shield which are comparable to the dimensions of the ILL nEDM experiment setup \cite{bib:baker, bib:knecht}.

%%%%%%%%%%%%%%%%%%%%%%%%%%%%%%%%%%%%%%%%%%%%%%%%%%%%%%%%%%%%%%%%%%%%55
% Taraneh
%%%%%%%%%%%%%%%%%%%%%%%%%%%%%%%%%%%%%%%%%%%%%%%%%%%%%%%%%%%%%%%%%%%%55
% make it clear this is for a spherical geometry that is close to the
%  ILL geometry.
%
Fig.~\ref{fig:Magnetic_Field} shows the central magnetic field $B$ as
a function of relative $\mu_r=\mu/\mu_0$ for coil radius 0.53~m inside a magnetic shield with the inner radius of $0.57$~m and a thickness of 1.5~mm which are
comparable to the dimensions of the ILL nEDM experiment
geometry~\cite{bib:baker,bib:knecht}.
\begin{figure}[h!]
\begin{center}
   \includegraphics[width=0.6\textwidth]{femm_and_calcs.pdf}
    \caption{Magnetic field as a function of relative magnetic
      permeability $\mu_r$ for geometries similar to the ILL nEDM
      experiment.  The dashed line is for an ideal spherical surface
      current of radius $0.53$~m inside a spherical shell of inner
      radius 0.57~m, thickness 1.5~mm.  Open and close circles are
      FEMM-based simulations of spherical and cylindrical geometries
      with similar dimensions, described in the text.  The coil
      currents have been arranged to give a $1~\mu$T field at the
      $\mu_r=20000$ point.}
    \label{fig:Magnetic_Field}
    \end{center}
\end{figure} 
For a 10\% change in relative $\mu$ (from 20000 to 22000), the
magnetic field changes by 0.7 nT.



\subsection{Magnetostatic Simulation Results}

Finite-element analysis simulations were conducted to analyze the
effect of discretizing the surface current, and to test the geometry
dependence of the sensitivity of the experiment to changes in $\mu$.
Two axially symmetric simulations were conducted using
FEMM~\cite{bib:femm}.  In the first simulation, the same spherical geometry was
used as for the analytical calculations.  However, the surface current
was discretized to 50 individual circular wire, inscribed onto a
sphere, and equally spaced vertically (i.e.~a discrete sin$\phi$
coil).  A square coil profile of side length 1~mm was used.  As shown
in Fig.~\ref{fig:Magnetic_Field}, this simulation gave good agreement
with the analytical calculation.

As an example of one additional axially symmetric geometry, a solenoid
within a cylindrical shield was simulated, with equal coil spacings.
In the limit of infinite $\mu$, the image currents in the end caps of
the shield are an infinite series of coils, giving an ideal infinite
solenoid with a uniform field.  Again, fifty discrete coils were
simulated, where the spacing from an end coil to the inner face of the
shield end-cap being half the inter-coil spacing, as appropriate to
generate the correct image currents in the infinite $\mu$ limit. 
%******************************************
% Taraneh : I commented out these parts because the conclusion is not consistent with the calculation of the reaction factor

%******************************************
 %As shown in Fig.~\ref{fig:Magnetic_Field}, the slope of $B(\mu)$ is
%somewhat steeper, and similar in magnitude to the spherical case.  We
%therefore conclude that the scale of the sensitivity of a generic nEDM
%experiment to global changes in the magnetic permeability is
%$\frac{\mu}{B}\frac{dB}{d\mu}\sim 0.01$.

In the spherical case the reaction factor is 1.39.  In the cylindrical
case the reaction factor is 1.35.  
%In general, it is expected that the
%larger the reaction factor, the large the sensitivity of the
%experiment to changes in the permeability of the shield.  This
%conclusion is also borne out by Fig.~\ref{fig:Magnetic_Shield}, in
%that the slope of the cylindrical simulation is steeper than that of
%the spherical simulation.

%\begin{itemize}
%\item I think it would be easy to provide results in axisymmetric
%  geometries from FEMM.
%\item If any result is available from OPERA, it could be included here.
%\item Results should be for a restricted set of possible nEDM coils.
%\item quote reaction factors?
%\item One graph?  % probably the same graph as above
%\end{itemize}

\subsection{Self-Shielded Coils}

A way to decouple the internal coil from the magnetic shield is to design a self-shielded
coil~\cite{bib:cpviolwithoutstrangeness,bib:someotherselfshieldedcoilpapers}.
In self-shielded coils, the return flux is provided by a second larger
coil, rather than through the permeable material of the magnetic
shield.  In a perfect self-shielded coil, the field at the position of
the magnetic shield would be zero, resulting in a reaction factor that
is identically unity.

Such coils would completely decouple the properties of the magnetic
shield from the homogeneity and stability of the coil itself.  Changes
in $\mu$ of the shield material would then have no impact on the nEDM
experiment.  Coils incorporating self-shielding in their design are
therefore an attractive option for nEDM
experiments~\cite{bib:cpviolwithoutstrangeness}.


%\begin{itemize}
%\item Explain principle (reaction factor = 1, or no field at shield so
%  no return flux)
%\item Simple analytic results - reaction factor is identically unity.
%\item FEMM/OPERA - demonstrate that sensitivity is reduced by factor
%  of XX for simple geometry.
%\item One graph, or include on previous graph?
%\end{itemize}

\subsection{Summary of $B_0(\mu)$}
For the shield-coupled coils the magnetic field in the region interior
to the coil is coupled to the properties of the magnetic shell which is caused
by the penetration of the magnetic field flux into the shell material. In a
typical nEDM experiment, the sensitivity of the internal magnetic
field to changes in $\mu$ is $\frac{\mu}{B}\frac{dB}{d\mu}\sim 0.01$.
The coupling can be much reduced by replacing the shield-coupled coils
with the self-shielded coils.

\section{Measurements of $\mu(T)$}

\subsection{Previous Measurements and their Relationship to nEDM Experiments}
The interest in the soft magnetic alloys and their properties have
been increased in the last few
decades~\cite{bib:pfeifer,bib:bozorth,bib:couderchon}. Couderchon et
al. measured the thermal variation of $\mu$ of two different Ni-Fe
alloys in a low field and they found $\frac{1}{\mu}\frac{d\mu}{dT}\simeq$1\%/K at room
temperature~\cite{bib:couderchon}. Based on the data sheet of a
similar material, $\mu$ would depend on temperature as
$\frac{1}{\mu}\frac{d\mu}{dT}\simeq$0.3-0.7\%/K~\cite{bib:kruppvdm}. These
are measurements of the temperature dependence of $\mu$ at frequencies
much higher than the nEDM experiment (50~Hz compared to 0.01~Hz). In
addition, the applied magnetic field $H$ and magnetic flux density $B$
in these experiments are not clear while the static magnetic field for
the nEDM experiment is about 1~$\mu$T.

In the present work, two different approaches to measure temperature
dependence of $\mu$ were pursued. To elucidate the behavior of our
passive mu-metal shields, experiments were done using two witness
cylinders, which are smaller open-ended cylinders (6" in length) made
of the same material and annealed at the same time as our big
prototype shields. We therefore expect they have the same magnetic
properties as the big passive shields, but they are much smaller and
easier to perform measurements with.
%\begin{itemize}
%\item Coucheron et al.?
%\item Krupp-VDM data sheet?
%\item Proper literature search on past measurements of $\mu(T)$.
%\item State caveats in order to use these:  dependence on B, H, f.
%\item State typical B, H, f that nEDM needs.  Motivates additional
 % measurements.
%\end{itemize}

\subsection{Axial Shielding Factor Measurements}
In the first approach, the witness cylinder was used as a magnetic
shield.  The general idea is to measure any changes in the shielding
factor corresponding to a temperature change of the witness cylinder.
A solenoid producing a known homogeneous magnetic field was placed
outside the witness cylinder as shown in Fig.~\ref{fig:geometry}.
The solenoid has 14 turns with XXX spacing between the wires. The radius and the half length of the solenoid is 17.44~cm while the radius and the half length of our most inner prototype passive shield is 18.44~cm. In this design, when the end caps of the innermost prototype passive shield are on, the this solenoid can be approximated by an infinite solenoid that produces a homogeneous magnetic field. 
The applied magnetic field was at the order of $\mu$T.
The solenoid current was varied sinusoidally at typically 0.01 to
10~Hz. 
A Bartington fluxgate magnetometer XXX (model number) measured the magnetic field at the center of the
witness cylinder while the temperature of the shield was measured.

% what was the solenoid field?  how homogeneous was the field?
% Discuss the coil design?  Typical coil current?
To increase the resolution of the measured signal from the fluxgate, a Signal Conditioning Unit (SCU) with a low pass filter set to 10~Hz (CHECK) and a gain set to typically 50 and higher was used. The conversion factor for these fluxgates is 7~$\mu$T/V.
The signal from the fluxgate was then demodulated by an SRS830 lock-in
% describe amplifier/filter of fluxgate (SCU), settings.
amplifier providing the in-phase and out-of-phase components of the
signal. The time constant on the lock-in was set to 3 seconds with 12~dB filter. The sensitivity settings was set by the amplitude of the measured signal.
% describe the lock-in settings

\begin{figure}[h!]
\begin{center}
   \includegraphics[width=0.8\textwidth]{geometry.PNG}
    \caption{This figure shows the setup of the axial shielding factor
      measurement. The witness cylinder with a radius of XXX and  a length of XXX is placed inside a
      solenoid with a radius and a half length of 17.44~cm. The axis of symmetry is along the $z$-axis. The windings
      of the solenoid is shown in red. The blue coil with one turn is
      coupled to the witness cylinder and it has a radius of XXX.  }
    \label{fig:geometry}
    \end{center}
\end{figure}
% Add dimensions to figure.  Remove box and white space around
% diagram.

An example of the typical data acquired is shown in
Fig. \ref{fig:B_vs_Temp}. Graph (a) shows the temperature of the
witness cylinder over four days. The temperature changes are about
3.5$^\circ$C and are caused by temperature variations of the room. The
magnetic field $B$ is the sum, in quadrature, of the in-phase and
out-of-phase components. For $f\lesssim 1$ Hz most of the measured
fluxgate signal is in the in-phase component. The magnetic field
tracks the temperature trend in an opposite way which is shown in
graph (b). The slope of the graph (c) has been calculated by using a
linear fit to the data. In general, we measured 0.3 \%/K $< \vert
\frac{1}{B} \frac{dB}{dT} \vert <$ 2.3 \%/K with the sign being
negative. The range also encompasses the typical deviation from a
linear $B(T)$ in the data (for example, those seen in
Fig.~\ref{fig:B_vs_Temp}(c).), and so we adopt this as an
uncertainty. Through a comprehensive set of systematic studies, the
source of the deviations from linearity could not be found. We suspect
it is due to a combination of mechanical stability and properties of
the material that cannot be embodied by a single temperature
slope. Thus the experiment tends to set a scale and sign for the
possible temperature dependence, rather than a value.
  \begin{figure}[h!]
\begin{center}
   \includegraphics[width=0.5\textwidth]{B_vs_T.png}
    \caption{These graphs show data of temperature dependence of the
      magnetic field taken over 80 hours. Graph (a) shows the
      temperature of the witness cylinder in $^\circ$C as a function
      of time. Graph (b) shows the magnetic field inside the witness
      cylinder at the center in $\mu$T versus time and graph (c) shows
      the magnetic field in $\mu$T versus temperature in
      $^\circ$C. The red line is the linear fit to data. At
      22$^\circ$C, $\vert \frac{1}{B}\vert \vert
      \frac{dB}{dT}\vert\simeq -1.5 \% /K.$ }
    \label{fig:B_vs_Temp}
     \vspace{-2.em}
    \end{center}
\end{figure} 

To relate the data to $\mu(T)$, finite element simulations in FEMM and
OPERA were performed.  From these simulations the ratio $\frac{\mu}{B}
\frac{dB}{d\mu}$ was calculated in a model where $\bold{B}=\mu
\bold{H}$ and $\mu$ is a constant i.e. the material is treated as
being linear. The term $\frac{\mu}{B}\frac{dB}{d\mu}\neq 0$ because
the witness cylinders are open ended. Combining the measurement and
the simulations, the temperature dependence of effective $\mu$ (at
$\mu$=20000) can be calculated by
\begin{equation}
\frac{1}{\mu}\frac{d\mu}{dT}= -\frac{\frac{1}{B}\frac{dB}{dT}}{\frac{\mu}{B}\frac{dB}{d\mu}}.
\end{equation}

\subsubsection{Systematic Errors}
%initial design stages

\paragraph{Initial Design}

The initial design of the $\mu(T)$ measurement included Tygon tubing wrapped around the witness cylinder in a spiral pattern to flow water and change the temperature of the witness cylinder. In those measurements, the amplitude of the signal was measured  from a Tektronix oscilloscope. The mechanical stability issues dominated the systematic uncertainties. When water was running, the flexibility of the tubing caused a movement in the witness cylinder. As an alternative, the tubes were replaced by copper tubing. In this case, the challenge was to create enough contact between the tubes and the witness cylinder which was not successful. In another design, a TEC was replaced with the tubing. The main issue with this design was that it did not provide enough cooling for the witness cylinder and also it was creating only local temperature changes on the witness cylinder. In addition, despite of using heat sinks, the heat created by the TEC itself made it very inefficient. To reduce the background noise in this measurement, all the components interior to the measurement volume should be non-magnetic. As a result, type T thermocouples were used which has copper and constantan conductors.


% T-type are less magnetic than K-type.  How much less and what's the
% systematic error, since they are not nonmagnetic?  I would say it is
% unknown.  Is there a way to assign a number to this?
%Taraneh: I couldn't find any number associated to this. I am not sure if the fluxgates are sensetive enough to measure it too.

\paragraph{Final Design}

%final design
The final design included the witness cylinder, which was placed with a holder inside the innermost prototype passive shield. The temperature variations of the witness cylinder was due to the temperature changes in the room which was following the on and off cycles of the building air conditioner. A Bartington magnetic field sensor (fluxgate) was hold inside the witness cylinder at the centre by plastic holder. The signal from the sensor was then demodulated by an SRS830 lock-in amplifier. To convert the final measured voltage from the sensor to $\mu$T, the sum in quadratures of the in-phase and out of phase components of the lock-in amplifier was multiplied by the scaling factor on the magnetic field sensor which is 7 $\mu$T/V.  
The mu-metal alloy has a thermal conductivity of 0.35 W/(cm$\cdot$K). Therefore, four thermocouples placed on different spots on the witness cylinder to correct for small temperature differences on each end of the witness cylinder.
In these measurements, the farther side of the passive shields had the end caps while the other side was left open to give access to the interior region. Based on the magnetic field maps inside the passive shields, for 10 $\mu$T applied magnetic field, the stability of the field was to 0.1 $\mu$T level and the magnetic field was even more stable closer to the closed side of the passive shields. Therefore the witness cylinder was pushed to the side with the end caps on the passive shields. 

% Another list
% - amplitude measurement on oscilloscope (no lock-in). (done)
% - speed of temperature change? temperature homogeneity?  (not here yet) 
% - other methods of cooling and why they didn't work. (not here yet)(done)
% - magnetic contamination by thermocouples (done)

% The above list maybe explains other possible variations on the
% method that you tried that all seemed to have worse systematic
% errors.

% Then, a list of systematic errors once you settled on the final
% measurement technique...

% - field profile/homogeneity (done)
% - mechanical stability (done)
% - thermal expansion (done)
% - temperature dependence of fluxgate and SCU (done)
% - temperature dependence of lock-in (done)
% - temperature dependence of coil resistance (not here yet)
% - null measurement (Cu, and why it doesn't give the whole story, not here yet) (done)
% - effect of endcaps? (influence of external fields, not here yet)
% - temperature dependence of reaction factor (negligible, not here yet)
% - degaussing (not here yet)
% - measurements on different witness cylinders? (not here yet)
% - vibrations of the building?
% - ...
%
% Another list is on the oldelog somewhere.  Please find it.
% The above list is longer and inclusive of the old list.
% See oldelog entry 141 in Magnetic Fields.
%
% This list of systematics is addressed now...

To eliminate the coupling of the field profile and mechanical stabilities, another coil was used which was coupled to the witness cylinder as shown in Fig.~\ref{fig:geometry}.  This also enabled us to
study the dependence of the field profile to $\mu(T)$. The result of this study did not show a significant change and data was similar to Fig. \ref{fig:B_vs_Temp} (c). Therefore, the systematic error due to the field profile is smaller than the unknown systematic errors of Fig. \ref{fig:B_vs_Temp} and similar figures.
% What is the value of the systematic error deduced from this study?
% I would say that the result of this study was that we got similar
% data to Fig. 3(c).  Therefore the systematic error due to field
% profile is smaller than the unknown systematic error of Fig. 3(c)
% and similar figures.  also partly addresses possible changes in
% reaction factor with temperature. (done)


%Temperature dependence of fluxgate and SCU
In these measurements, a Mag-03IEL70 Bartington magnetic field sensor
which has individual sensor elements and a Mag-03SCU signal
conditioning unit were used.  The magnetic field sensor is a low noise
sensor with a range of $\pm$70 $\mu$T and it has a scaling temperature
coefficient of 15 ppm/K.  This is insignificant compared to the
temperature slopes typically observed.


As another mechanical stability study, the movement of the Bartington
fluxgate flying lead due to thermal expansion was estimated. If the
fluxgate flying lead move about 1 mm normal to its axis of symmetry
which is parallel to the axis of the witness cylinder, the magnetic
field will change about 30 ppm/K over 20~K temperature changes.  The
thermal expansion of the mu-metal cylinder is at the order of 10~ppm/K
\cite{kruppvdm}.  The SRS 830 lock-in amplifier has 50~ppm/K amplitude
stability~\cite{bib:lockin}.  In summary, all systematic changes with
temperature from these effects are significantly smaller than the
measured temperature slopes.

%We did this for the transformer technique
The temperature dependence of the coil resistance was also measured by
measuring the voltage across a temperature controlled 1~$\Omega$
resistor.
%The results?

%Taraneh: I calculated from the long summary
The stability of the system was tested by replacing the mu-metal
witness cylinder with a similar copper cylinder. For all those
measurements the stability was $<0.1$\%/K. Although these type of
measurements help to find an error bar on the stability of the system
in general, they do not help to find the sources of the systematic
uncertainty.
% note caveats - the field is quite different depending on the settings.



% Two possibilities:

% 1. unknown systematic error(s)

% 2. complicated material properties that don't follow a straight line
% could depend on material history themselves.

% Therefore we state a range which should be indicative of the scale
% of possible temperature dependence.


In this method it was found that 0.6\%/K
$\lesssim\frac{1}{\mu}\frac{d\mu}{dT}\lesssim 2.3\%$/K. Although for most of the measurements the general trend of $B(T)$ graphs was consistent, the shape and positions of the minor loops were changing due to unidentified sources of systematic errors and these are all included in the reported error bar of this measurement. A reasonable hypothesis might be different properties of each witness cylinder related to different annealing processes.



%\begin{itemize}
%\item Describe experimental setup and important considerations
%  (e.g. relationship of data to effective $\mu$)
%\item Explain B, H, f, and dominant systematic effects.
%\item One figure of experimental setup?
%\item One data graph?
%\item State overall result and systematic error.
%\end{itemize}

\subsection{Transformer Core Measurements}
In second approach, the witness cylinder was used as the core of a
transformer. Two coils (primary and sense coils) were wound on the
witness cylinder. The primary coil generated an AC magnetic field
using the internal oscillator of an SRS830 lock-in amplifier while the
sense coil measured the sense voltage and was picking up the magnetic
field flux produced by the primary coil. This experiment was done at 1
Hz and as small as possible $H$,
% state typical value
to measure the slope of the minor $B-H$ loops near the origin of the
$B-H$ space. In lower frequencies, noise dominated the actual signal.
% why?
The temperature of the core was measured concurrently as the sense
voltage using non-magnetic type T thermocouples.  The measured voltage
induced on the sense coil, is proportional to $\mu$.  The measurements
show a range of 0.1\%/K to 2\%/K for $\frac{1}{\mu}\frac{d\mu}{dT}$,
again assuming the material to be linear.

In all transformer measurements, the sense voltage from the sense coil
was demodulated by a lock-in amplifier. In
Fig.~\ref{fig:data_and_simulation} the in-phase component $Y$ and
out-of-phase component $X$ of the sense signal as a function of
applied $H$ field at 1 Hz are shown. In the ideal case where the
material is linear, the $X$ component should be zero. The ratio of the
$X/Y$ depends on both the applied frequency and the position on the
$B-H$ region. We were interested in regions where $\vert Y \vert >
\vert X \vert $ in which the material behaves as being linear and the
$\vert H \vert $ values correspond to $\vert X \vert < \vert X_{max}
\vert$ while still being able to get a good signal to noise ratio.


A concern in both measurements is that, the material is not
linear. There are other effects such as hysteresis, saturation, eddy
currents and skin depth which all contribute to the measured signal
and so it is not possible to embody the results as a single parameter.
To check the sensitivity of the sense coil demodulated voltage to
these effects, a theoretical model of the hysteresis (Jiles-Atherton
model) was used \cite{bib:jiles}, see Fig.
\ref{fig:data_and_simulation}.

The key parameters of Jiles-Atherton model are saturation
magnetization $M_s$, mean field parameter $\alpha$ which represents
interdomain coupling, $a$ with the dimensions of magnetic field which
characterizes the shape of anhysteretic magnetization, bulk
magnetization $M$ and the magnetic field $H$. A shown in
Fig. \ref{fig:data_and_simulation}, trends in the measurements and
simulations are fairly consistent.

Some measurements regarding degaussing have been done to study how
removing magnetization effects the sense voltage. Degaussing
(idealizing) magnetic shields means to saturate the shields by
applying a very large magnetic field and then slowly ramping this
applied field down to zero.
% State how done.
In most cases, degaussing helped to
increase the signal and hence $\mu$.
% But it didn't change its temperature dependence significantly.  Poor
% degaussing could result in initial slopes that were large for a
% while?

\begin{figure}[h!]
\begin{center}
   \includegraphics[width=0.5\textwidth]{data_and_simulation3.PNG}
    \caption{These graphs represent the sense voltage as a function of applied $H$ field at 1 Hz. Graph (a) shows the out-of-phase component $X$ of the measured sense voltage as well as the simulation. Graph (b) shows the in-phase component $Y$ of the signal and its simulation.}
    \label{fig:data_and_simulation}
    \end{center}
\end{figure} 

\subsubsection{Systematic Errors}

% Need list
% Please place here
%
% unique ones:
% - motion of wires
% - degaussing, how done, and general observations
% - resistance of wires
% - different numbers of turns on primary/secondary
% - in-phase/out-of-phase components, how small should in-phase be?
% - ...


The dominant source of systematic uncertainty in this method arises
from the properties of witness cylinders. Saturation and magnetization
of the witness cylinders and the nonlinearity of $/mu$ in general had
a considerable effect on the measured signal. Despite of the
consistent general trend of $B(T)$, different witness cylinders tend
to behave differently under similar conditions. The applied voltage on
the primary coil was stable and was measured across a 1~$\Omega$
temperature controlled resistor. To reduce the background noise and
thermally isolate the witness cylinder, some measurements were done by
placing the witness cylinder inside the big passive shields. The
results did not show a significant change.

%\begin{itemize}
%\item Describe experimental setup and important consdierations (e.g. relationship of data to effective $%\mu$)
%\item Explain B, H, f, and dominant systematic effects.
%\item Understanding in terms of Jiles-Atherton.  Complications that
 % this does not translate well into ``$\mu$''.
%\item One data graph?  (Perhaps the Jiles-Atherton one?)
%\item State overall result and systematic error.
%\end{itemize}
%%%%%%%%%%%%%%%%%%%%%%%%%%%%%%%%%%%%%%%%%%%%%%%%%%%%%%%%%%%%%%%%%%%%%%
% Taraneh, can you write text up to this point?
%%%%%%%%%%%%%%%%%%%%%%%%%%%%%%%%%%%%%%%%%%%%%%%%%%%%%%%%%%%%%%%%%%%%%%

\subsection{Summary of $\mu(T)$}
%Insert Summary Here.
The overall error bar in $\mu$ from both measurement techniques is 0.1\%/K $\leq \frac{1}{\mu}\frac{d\mu}{dT} \leq$ 2.3 \%/K and it is consistent with previous data.
Comparing to the static nEDM experiment with $f < 0.01$ Hz, these measurements are mostly done at frequencies close to 1 Hz. Hence, the effective $\mu$ extracted from these type of experiments give
only a qualitative information on the actual nEDM experiment situation and to reach higher accuracy the nEDM experiment has to be built. 
\section{Summary and Conclusions}
The magnetic field internal to the passive shields as a function of $\mu$ for spherical geometry with parameters close to the ILL nEDM experiment has been analytically calculated and compared to axis-symmetric simulations with similar parameters for both spherical and cylindrical geometries. For spherical geometry $\frac{dB_0}{d\mu}\sim 4 \times 10^{-13}$ and it can be much reduced for the shield coupled coils.
The temperature dependence of $\mu$ has been measured by two different techniques using open-ended mu-metal cylinders with 6 inches in length. The overall result gives 0.1\%/K$\frac{1}{\mu}\frac{d\mu}{dT}$2.3 \%/K. Because of the different values in $f$, $B$ and $H$ fields as opposed to the nEDM experiments, these  measurements help to set the specifications for the actual nEDM setup. The nonlinearities in the witness cylinder material such as hysteresis dominated the systematic uncertainties at both measurements and so it was difficult to achieve consistent sub-percent determinations although the sign and the magnitude of the problem was in agreement.
To achieve the required 1~pT stability in the internal field, the temperature should be controlled to 0.1 K level.
The next step is to measure $\mu(T)$ for the nEDM setup with different coil geometries to achieve the best stability.



%\begin{itemize}
%\item We determined $dB_0/d\mu$.  For reasonable parameters its $\sim$ XXX.  For self-shielded it's reduced by a factor of XXX for realistic windings.
%\item We measured $d\mu/dT$.  We constrained it to be in the range 0.1
 % to 2.3 percent per Kelvin.  Dominant errors are XXX for each method.
  %Neither method is for correct B, H, f.  Both methods are very
  %difficult to achieve consistent sub-percent determinations, but tend
  %to agree on sign and magnitude of problem.
%\item Results imply temperature stability to XXX Kelvin for stability
 % goal of XXX pT.  Show that this could be a dominant issue for
  %stability for future experiments.  Not addressed by many other
  %measurements which tend to focus on stability at zero field or
  %overall stability in full EDM experiment.
%\item Future work: build full EDM experiment.  Capability of various
 % internal coil geometries and best achievable temperature stability
  %are both important to success.
%\end{itemize}

%\section{To Do List}

%\begin{enumerate}
%\item Collect graphs and figures.
%\begin{enumerate}
%\item Analytical comparison reaction factor vs. $\mu$ for
 % sphere and/or cylinder with reasonable geometry/$\mu$.
%\item Same or additional comparison for more realistic geometry in
%  simulation.  May include self-shielded geometry.
%\item Axial shielding factor figure of setup.
%\item Axial shielding factor data.
%\item Transformer core Jiles-Atherton comparison?
%\end{enumerate}
%\item Write.
%\end{enumerate}

\section*{References}

\bibliography{mybibfile}
\begin{thebibliography}{00}

\bibitem{bib:nedm1} S. N. Balashov {\it et al.}, arXiv:0709.2428.

\bibitem{bib:nedm2} A. P. Serebrov {\it et al.}, JETP Lett. {\bf 99}, 4
  (2014).

\bibitem{bib:nedm2.5} A. P. Serebrov {\it et al.}, Physics Procedia {\bf
  17}, 251 (2011).

\bibitem{bib:nedm3} K. Kirch, AIP Conf. Proc., Vol. 1560, pp. 90-94
  (2013).

\bibitem{bib:nedm3.5} C. A. Baker, {\it et al.}, Physics Procedia {\bf
  17}, 159 (2011).

\bibitem{bib:nedm4} Y. Masuda, K. Asahi, K. Hatanaka, S.-C. Jeong,
  S. Kawasaki, R. Matsumiya, K. Matsuta, M. Mihara, and Y. Watanabe,
  Phys. Lett. A {\bf 376}, 1347 (2012).

\bibitem{bib:nedm5} I. Altarev, {\it et al.}, Nuovo Cim. C {\bf
  35}, 122 (2012).

\bibitem{bib:nedm6} R. Golub and S. K. Lamoreaux, Phys. Rept.  {\bf
  237}, 1 (1994).

\bibitem{bib:nedm6.5} T. M. Ito (the nEDM collaboration),
  J. Phys. Conf. Ser. {\bf 69} 012037, 2007.

\bibitem{bib:baker} C. A. Baker, {\it et al.}, Phys. Rev. Lett. {\bf
  97}, 131801 (2006).

\bibitem{bib:brys} T. Bry\'s, {\it et al.}, Nucl. Instrum. Meth. A
  {\bf 554}, 527 (2005).

\bibitem{bib:afach} S. Afach, {\it et al.}, J. Appl. Phys. 116, 084510 (2014).

\bibitem{bib:fierlingerroom} I. Altarev, {\it et al.}
  Rev. Sci. Instrum. {\bf 85}, 075106 (2014).

\bibitem{bib:sturmthesis} M. Sturm, Masterarbeit, T.U. Muenchen (2013).

\bibitem{bib:patton} B. Patton, E. Zhivun, D. C. Hovde, and D. Budker,
  Phys. Rev. Lett. 113, 013001 (2014).
\bibitem{bib:couderchon} G. Couderchon, J. F. Tiers, J. Magn. Magn. Mat. 26(1):196-214 (1982).

\bibitem{bib:bidinosti} C. P. Bidinosti {\it et al.}, AIP Advances 4, 047135 (2014).

\bibitem{bib:smythe} W. R. Smythe, McGraw Hill (1950).

\bibitem{bib:ferraro} V. C. A. Ferraro, Athalone Press (1967).

\bibitem{bib:gupta} Kiran Gupta, K. K. Raina, S. K. Sinha, J. Alloys Compd, 429(1):357-364 (2007).

\bibitem{bib:knecht} A. Knecht, Dissertation, UZH Zurich (2009).

\bibitem{bib:kruppvdm} KruppVDM Magnifer 7904, MDS no. 9004 (2000).

\bibitem{bib:jiles} D. C. Jiles, D. L. Atherton,  J. Magn. Magn. Mat. 61(1-2):48-60, (1986).

\bibitem{bib:pendlebury} J. M. Pendlebury {\it et al.}, Phys. Rev. D 92, 092003 (2015).

\bibitem{bib:pfeifer} F. Pfeifer, C. Radeloff, J. Magn. Magn. Mat. 19 (1980) 190-207.

\bibitem{bib:thiel} F. Thiel {\it et al.}, Rev. Sci. Instrum. 78, 035106 (2007).

\bibitem{bib:altarev2014} I. Altarev {\it et al.},Rev. Sci. Instrum. 85, 075106 (2014).

\bibitem{bib:altarev2015} I. Altarev {\it et al.}, J. Appl. Phys. 117, 233903 (2015).

\bibitem{bib:voigt} J. Voigt {\it et al.}, Metrol. Meas. Syst. 20, 239 (2013).

\bibitem{bib:bozorth} R. M. Bozorth, Ferromagnetism, IEEE Press (1993).

\bibitem{bib:lockin} Stanford Reserach Systems, DSP lock-in amplifier, Model SR830 (2011)
\end{thebibliography}

\end{document}
