\documentclass[review]{elsarticle}
\usepackage{hyperref,amssymb,graphicx,booktabs}

%\modulolinenumbers[5]
\journal{Nuclear Instruments and Methods in Physics Research A}

%%%%%%%%%  use this for permeability of free space %%%%%%%%%%%%%%%
\newcommand{\muo}{ \mu_{\scriptscriptstyle 0}}




\begin{document}

\begin{frontmatter}

\title{Sensitivity of Fields Generated within Magnetically Shielded
  Volumes to Changes in Magnetic Permeability}

\author[manitoba]{T.~Andalib}
\author[winnipeg,manitoba]{J.W.~Martin\corref{mycorrespondingauthor}}
\cortext[mycorrespondingauthor]{Corresponding author}
\ead{j.martin@uwinnipeg.ca}
\author[winnipeg,manitoba]{C.P.~Bidinosti}
\author[winnipeg,manitoba]{R.R.~Mammei}
\author[winnipeg,manitoba]{B.~Jamieson}
\author[manitoba]{M.~Lang}
\author[triumf]{T.~Kikawa}


\address[winnipeg]{Physics Department, The University of Winnipeg, 515 Portage Avenue, Winnipeg, MB, R3B 2E9, Canada}
\address[manitoba]{Department of Physics and Astronomy, University of Manitoba, Winnipeg, MB R3T 2N2, Canada}
\address[triumf]{TRIUMF, 4004 Wesbrook Mall, Vancouver, BC V6T 2A3, Canada}


\begin{abstract}
Future experiments seeking to measure the neutron electric dipole
moment (nEDM) require stable and homogeneous magnetic fields.
Normally these experiments use a coil internal to a passively
magnetically shielded volume to generate the magnetic field.  The
stability of the magnetic field generated by the coil within the
magnetically shielded volume may be influenced by a number of factors.
The factor studied here is the dependence of the internally generated
field on the magnetic permeability $\mu$ of the shield material.  We
provide measurements of the temperature-dependence of the permeability
of the material used in a set of prototype magnetic shields, using
experimental parameters nearer to those of nEDM experiments than
previously reported in the literature.  Our measurements imply a range
of $\frac{1}{\mu}\frac{d\mu}{dT}$ from 0-2.7\%/K. Assuming typical
nEDM experiment coil and shield parameters gives
$\frac{\mu}{B_0}\frac{dB_0}{d\mu}=0.01$, resulting in a temperature
dependence of the magnetic field in a typical nEDM experiment of
$\frac{dB_0}{dT}=0-270$~pT/K for $B_0=1~\mu$T.  The results are useful
for estimating the necessary level of temperature control in nEDM
experiments.
\end{abstract}

\begin{keyword}
Magnetic Shielding \sep Neutron Electric Dipole Moment \sep Magnetic Field Stability
\end{keyword}

\end{frontmatter}

%\linenumbers

\section{Introduction}

The next generation of neutron electric dipole moment (nEDM)
experiments aim to measure the nEDM $d_n$ with proposed precision
$\delta d_n\lesssim
10^{-27}~e\cdot$cm~\cite{bib:nedm2,bib:nedm2.5,bib:nedm3,bib:nedm3.5,bib:nedm5,bib:nedm6,bib:nedm6.5,bib:nedmtriumf}.
In the previous best experiment \cite{bib:baker,bib:pendlebury} which
discovered $d_n<3.0\times 10^{-26}~e\cdot$cm (90\% C.L), effects
related to magnetic field homogeneity and instability were found to
dominate the systematic error.  A detailed understanding of passive
and active magnetic shielding, magnetic field generation within
shielded volumes, and precision magnetometry is expected to be crucial
to achieve the systematic error goals for the next generation of
experiments.  Much of the research and development efforts for these
experiments are focused on careful design and testing of various
magnetic shield geometries with precision
magnetometers~\cite{bib:brys,bib:afach,bib:fierlingerroom,bib:sturmthesis,bib:patton}.

In nEDM experiments, the spin-precession frequency $\nu$ of neutrons
placed in static magnetic $B_0$ and electric $E$ fields is measured.
The measured frequencies for parallel $\nu_+$ and antiparallel $\nu_-$
relative orientations of the fields is sensitive to the neutron
electric dipole moment $d_n$
\begin{equation}
h\nu_\pm=2\mu_nB_0\pm 2d_nE
\end{equation}
where $\mu_n$ is the magnetic moment of the neutron.

A problem in these experiments is that if the magnetic field $B_0$
drifts over the course of the measurement period, it degrades the
statistical precision with which $d_n$ can be determined.  If the
magnetic field over one measurement cycle is determined to $\delta
B_0=10$~fT, it implies an additional statistical error of $\delta
d_n\sim 10^{-26}~e\cdot$cm (assuming an electric field of $E=10$~kV/cm
which is reasonable for a neutron EDM experiment).  Over 100 days of
averaging, this would make a $\delta d_n\sim 10^{-27}~e\cdot$cm
measurement possible.  Unfortunately the magnetic field in the
experiment is never stable to this level.  For this reason,
experiments use a comagnetometer and/or surrounding atomic
magnetometers to measure and correct the magnetic field to this
level~\cite{bib:baker,bib:brys,bib:afach}.  Drifts of 1-10~pT in $B_0$
may be corrected using the comagnetometer technique, setting a goal
magnetic stability for the $B_0$ field generation system in a typical
nEDM experiment.

In such experiments, typically $B_0=1~\mu$T is used to provide the
quantization axis for the ultracold neutrons.  The $B_0$ magnetic
field generation system typically includes a coil placed within a
passively magnetically shielded volume.  The passive magnetic shield
is generally composed of a multi-layer shield formed from thin shells
of material with high magnetic permeability (mu-metal).  The outer
layers of the shield are normally
cylindrical~\cite{bib:nedm2,bib:nedm3.5} or form the walls of a
magnetically shielded room~\cite{bib:altarev2014,bib:altarev2015}.
The innermost magnetic shield is normally a specially shaped shield,
where the design of the coil in relation to shield is carefully taken
into account to achieve adequate homogeneity
\cite{bib:baker,bib:nedm3,bib:nedm5}.

Mechanical and temperature changes of the passive magnetic
shielding~\cite{bib:voigt,bib:thiel}, and the degaussing
procedure~\cite{bib:thiel,bib:altarev2015,bib:fierlinger2016} (also
known as demagnetization, equilibration, or idealization), affect the
stability of the magnetic field within magnetically shielded rooms.
Active stabilization of the background magnetic field surrounding
magnetically shielded rooms can also improve the internal
stability~\cite{bib:voigt,bib:afach,bib:franke}.  The current supplied
to the $B_0$ coil is generated by an ultra-stable current
source~\cite{bib:brys}. The coil must also be stabilized mechanically
relative to the magnetic shielding.

One additional effect, which is the subject of this paper, relates to
the fact that the $B_0$ coil in most nEDM experiments is magnetically
coupled to the innermost magnetic shield.  If the magnetic properties
of the innermost magnetic shield change as a function of time, it then
results in a source of instability of $B_0$.  In the present work, we
estimate this effect and characterize one possible source of
instability: changes of the magnetic permeability $\mu$ of the
material with temperature.

While the sensitivity of magnetic alloys to temperature variations has
been characterized in the past~\cite{bib:couderchon,bib:kruppvdm}, we
sought to make these measurements in regimes closer to the operating
parameters relevant to nEDM experiments.  For these alloys, it is also
known that the magnetic properties are set during the final annealing
process~\cite{bib:gupta,bib:bozorth,bib:kruppvdm}.  In this spirit we
performed our measurements on ``witness'' cylinders, which are small
open-ended cylinders made of the same material and annealed at the
same time as other larger shields are being annealed.

The paper proceeds in the following fashion:
\begin{itemize}
\item The dependence of the internal field on magnetic permeability of
  the innermost shielding layer for a typical nEDM experiment geometry
  is estimated using a combination of analytical and finite element
  analysis techniques.  This sets a scale for the stability problem.
\item New measurements of the temperature dependence of the magnetic
  permeability are presented.  The measurements were done in two ways
  in order to study a variety of systematic effects that were
  encountered.
\item Finally, the results of the calculations and measurements are
  combined to provide a range of temperature sensitivities that takes
  into account sample-to-sample and measurement-to-measurement
  variations.
\end{itemize}


\section{Sensitivity of Internally Generated Field to Permeability of the Shield $B_0(\mu)$}

\subsection{Analytical Calculations in Spherical (and Cylindrical) Geometry}


%\begin{itemize}
%\item State basic physics of problem, define reaction factor, note
%  return flux through shield, etc.
%\item State formulae and results for reasonable parameters.
%\item Possibly one graph of $B_0(\mu)$ with both geometries.
%\end{itemize}

The presence of a coil inside the innermost passive shield turns the
shield into a return yoke and it is due to the penetration of the
magnetic field flux into the magnetic shield.  The ratio of the
magnetic field inside the coil in the presence of the magnetic shield
to that of the coil in free space is called the reaction
factor~\cite{bib:urankar, bib:bidinosti}.
Normally the reaction factor is larger than unity for spherical and
cylindrical geometries.  The key issue of interest for this work is
the dependence of the reaction factor o
n $\mu$.  This factor can be
calculated analytically for spherical and cylindrical
geometries~\cite{bib:bidinosti,bib:urankar}.  Although
the dependence of the reaction factor on $\mu$ is rather weak for
these geometries, the constraints on $B_0$ stability are very
stringent.  Thus a small change in the magnetic properties of the
innermost shield can result in an unacceptably large change in $B_0$.

%From Refs. \cite{bib:smythe, bib:ferraro} for a zonal surface current
%\begin{equation}
%\bold{F}=- \sum_{n=1} ^{\infty} \frac{C_n}{\ell} P^1 _n (u) \hat{\phi}
%\end{equation}
%bound on a sphere with radius $\ell$ the magnetic field at $r < \ell$ and $r>\ell$ is calculated.
%The case of an ideal spherical surface current ($n=1$) inside a spherical shell with inner radius $a$ and outer radius $b$ has been analytically calculated considering the following boundary conditions

%\begin{equation}
%\frac{1}{\mu_0} B_{\theta} (a^-) = \frac{1}{\mu} B_{\theta}(a^+)
%\end{equation}
%\begin{equation}
%\frac{1}{\mu}B_{\theta}(b^-)=\frac{1}{\mu_0}B_{\theta}(b^+).
%\end{equation}
%Fig. \ref{fig:Magnetic_Field} shows the magnetic field $B$ as a function of relative $\mu$ for $\ell=0.53$ m, $a=0.57$ m and a thickness of 1.5 mm for the shield which are comparable to the dimensions of the ILL nEDM experiment setup \cite{bib:baker, bib:knecht}.

Fig.~\ref{fig:Magnetic_Field} shows the central magnetic field $B$ as
a function of relative $\mu_r=\mu/\mu_0$ for coil radius 0.53~m inside
a magnetic shield with the inner radius of $0.57$~m and a thickness of
1.5~mm.  The dimensions have been selected to be comparable to the
dimensions of the ILL nEDM experiment
geometry~\cite{bib:baker,bib:knecht}.
\begin{figure}[h!]
\begin{center}
   \includegraphics[width=0.6\textwidth]{femm_and_calcs.pdf}
    \caption{Magnetic field as a function of relative magnetic
      permeability $\mu_r$ for geometries similar to the ILL nEDM
      experiment.  The dashed line is for an ideal spherical surface
      current of radius 0.53~m inside a spherical shell of inner
      radius 0.57~m, thickness 1.5~mm.  Open and close circles are
      FEMM-based simulations of spherical and cylindrical geometries
      with similar dimensions, described in the text.  The coil
      currents have been arranged to give a $1~\mu$T field at the
      $\mu_r=20,000$ point.}
    \label{fig:Magnetic_Field}
    \end{center}
\end{figure} 
For a 10\% change in relative $\mu_r$ (from 20000 to 22000), the
magnetic field changes by 0.7~nT.



\subsection{Magnetostatic Simulation Results \label{sec:femm}}

Finite-element analysis simulations were conducted to analyze the
effect of discretizing the surface current, and to test the geometry
dependence of the sensitivity of the experiment to changes in $\mu$.
Two axially symmetric simulations were conducted using
FEMM~\cite{bib:femm}.  In the first simulation, the same spherical
geometry was used as for the analytical calculations.  However, the
surface current was discretized to 50 individual circular wires,
inscribed onto a sphere, and equally spaced vertically (i.e.~a
discrete cos$\theta$ coil).  A square sire profile of side length 1~mm
was used.  As shown in Fig.~\ref{fig:Magnetic_Field}, this simulation
gave good agreement with the analytical calculation.

As an example of one additional axially symmetric geometry, a solenoid
within a cylindrical shield was simulated, with equal coil spacings.
In the limit of infinite $\mu$, the image currents in the end caps of
the shield are an infinite series of coils, giving an ideal infinite
solenoid with a uniform field.  Again, fifty discrete coils were
simulated, where the spacing from an end coil to the inner face of the
shield end-cap being half the inter-coil spacing, as appropriate to
generate the correct image currents in the infinite $\mu$ limit. 

As shown in Fig.~\ref{fig:Magnetic_Field}, the slope of $B(\mu)$ is
somewhat steeper, and similar in magnitude to the spherical case.  We
therefore estimate that the scale of the sensitivity of a generic nEDM
experiment to global changes in the magnetic permeability is
$\frac{\mu}{B}\frac{dB}{d\mu}\sim 0.01$.

In the spherical case the reaction factor is 1.39, meaning that the
field generated by the spherical coil is amplified by this
multiplicative factor by the presence of the magnetic shield (flux
return).  In the cylindrical case the reaction factor is 1.35.  If the
reaction factor is closer to unity, the calculated
$\frac{\mu}{B}\frac{dB}{d\mu}$ will be smaller, and the EDM experiment
less sensitive to changes in $\mu$.  The results in the two geometries
show that there is also a weaker geometry dependence to this
statement.


%\begin{itemize}
%\item I think it would be easy to provide results in axisymmetric
%  geometries from FEMM.
%\item If any result is available from OPERA, it could be included here.
%\item Results should be for a restricted set of possible nEDM coils.
%\item quote reaction factors?
%\item One graph?  % probably the same graph as above
%\end{itemize}

\subsubsection{Field within the passive flux return for EDM experiments}

For a high-$\mu$ innermost shield, the magnetic field lines emanating
from the coil all return through the shield.  This principle can be
used to estimate the magnetic field internal to the material, and in
our studies gave good agreement with FEA-based simulations.  For the
cylindrical geometry described in Sec.~\ref{sec:femm}, the $B$ field
is largest in the side walls of the cylindrical flux return, attaining
a maximum value of 170~$\mu$T.  For $\mu_r$=20,000 (a typical
realistic value for the DC value of $\mu_r$ in these shields), the $H$
field is 0.007~A/m.  These values are weakly dependent on $\mu_r$.
They set a scale for the values of $B$ and $H$ which we sought to
replicate in our experiments reported in Section~\ref{sec:tdep}.

% Jeff and Taraneh to do: confirm by checking against FEMM.  Possibly
% add more numbers for spherical geometry.


\subsection{Self-Shielded Coils}

A way to decouple the internal coil from the magnetic shield is to
design a self-shielded
coil~\cite{bib:cpviolwithoutstrangeness,bib:someotherselfshieldedcoilpapers}.
In self-shielded coils, the return flux is provided by a second larger
coil, rather than through the permeable material of the magnetic
shield.  In a perfect self-shielded coil, the field at the position of
the magnetic shield would be zero, resulting in a reaction factor that
is identically unity.

% Taraneh says: Ask Chris for the reference.
Such coils would completely decouple the properties of the magnetic
shield from the homogeneity and stability of the coil itself.  Changes
in $\mu$ of the shield material would then have no impact on the nEDM
experiment.  Coils incorporating self-shielding in their design are
therefore an attractive option for nEDM
experiments~\cite{bib:cpviolwithoutstrangeness}.


%\begin{itemize}
%\item Explain principle (reaction factor = 1, or no field at shield so
%  no return flux)
%\item Simple analytic results - reaction factor is identically unity.
%\item FEMM/OPERA - demonstrate that sensitivity is reduced by factor
%  of XX for simple geometry.
%\item One graph, or include on previous graph?
%\end{itemize}

\subsection{Summary of $B_0(\mu)$ \label{sec:theory_summary}}

For shield-coupled coils, the magnetic field in the region interior to
the coil is coupled to the properties of the magnetic shell which is
caused by the penetration of the magnetic field flux into the shell
material. In a typical nEDM experiment, the sensitivity of the
internal magnetic field to changes in $\mu$ is
$\frac{\mu}{B}\frac{dB}{d\mu}\sim 0.01$.  The coupling can be reduced
considerably by using a self-shielded coil design, which does not
couple strongly to the innermost magnetic shield.

The field generated in the innermost magnetic shield by a
shield-coupled coil is typically $B=170~\mu$T, corresponding to
$H=0.007$~A/m.  The field in the nEDM measurement volume, as well as
in the magnetic shield, must be stable for periods of typically
hundreds of seconds (corresponding to frequencies $<<0.01$~Hz).  This
sets the relevant measurement scales for magnetic properties that
would be most relevant to nEDM experiment.




\section{Measurements of $\mu(T)$\label{sec:tdep}}

\subsection{Previous Measurements and their Relationship to nEDM Experiments\label{sec:previousmeasurement}}

Previous measurements of the temperature dependence of the magnetic
properties of high-permeability alloys have been summarized in
Refs.~\cite{bib:couderchon,bib:bozorth,bib:pfeifer}.  These
measurements are normally conducted using a sample of the material to
create a toroidal core, where a thin layer of the material is used in
order to avoid eddy-current and skin-depth
effects~\cite{bib:pfeifer,bib:kruppvdm}.  A value of $\mu$ is
determined by dividing the amplitude of the sensed $B_m$-field by the
amplitude of the driving AC $H_m$-field (similar to the method
described in Section~\ref{sec:transformer}).  Normally the frequency
of the $H_m$-field is 50 or 60~Hz.  The value of $\mu$ is then quoted
either at or near its maximum attainable value by adjusting the
amplitude of $H_m$.  Depending on the details of the $B_m-H_m$ curve
for the material in question, this normally means that $\mu$ is quoted
for the amplitude of $H_m$ being at or near the coercivity of the
material~\cite{bib:couderchon,bib:kruppvdm}, resulting in large values
up to $\mu_r=4\times 10^5$.

It is well known that $\mu$ measured in this fashion for toroidal,
thin metal wound cores depends on the annealing process used for the
core.  There is a particularly strong dependence on the take-out or
tempering temperature after the high-temperature portion of the
annealing process has been
completed~\cite{bib:pfeifer,bib:kruppvdm,bib:couderchon}.  Such
studies normally suggest a take-out temperature of 490-500$^\circ$C.
This ensures that the large $\mu_r=4\times 10^{5}$ is furthermore
maximal at room temperature.  Slight variations around room
temperature, and assuming the take-out temperature is not controlled
to better than a degree, imply a scale of possible temperature
variation of $\mu$ of approximately
$\left|\frac{1}{\mu}\frac{d\mu}{dT}\right|\simeq 0.3$-1\%/K at room
temperature~\cite{bib:couderchon,bib:kruppvdm}.

A challenge in applying these results to temperature stability of nEDM
experiments is that, when used as DC magnetic shielding, the
high-permeability alloys are usually operated for significantly
different parameters ($B_m$, $H_m$, and frequencies).

For example, when used in a shielding configuration, the effective
permeability is often measured to be typically $\mu_r=20,000$ rather
than $4\times 10^5$.  This arises in part because $H_m$ is well below
the DC coercivity.  As noted in Section~\ref{sec:calculation}, a more
appropriate $H_m$ for the innermost magnetic shield of an nEDM
experiment is $<0.007$~A/m, whereas the coercivity is
$H_c=0.4$~A/m~\cite{bib:kruppvdm}.  The frequency dependence of the
measurements could also be an issue.  Typically, nEDM experiments are
concerned with slow drifts at $<0.01$~Hz timescales whereas the
previously reported $\mu(T)$ measurements are performed in an AC mode
at 50-60~Hz.


The goal of our experiments was to develop techniques to characterize
the material properties of our own magnetic shields post-annealing, in
regimes more relevant to nEDM experiments.


We created a prototype passive magnetic shield system in support of
this and other precision magnetic field research for the future nEDM
experiment to be conducted at TRIUMF.  The shield system is a
four-layer mu-metal shield formed from nested right-circular
cylindrical shells with endcaps.  The inner radius of the innermost
shield is 18.44~cm, equal to its half-length. The radii and
half-lengths of the progressively larger outer shields increase
geometrically by a factor of 1.27.  Each cylinder has two endcaps
which possess a 7.5~cm diameter central hole.  A stove-pipe of length
5.5~cm is placed on each hole was designed to minimize leakage of
external fields into the progressively shielded inner volumes.  The
design is similar to another smaller prototype shield discussed in
Ref.~\cite{bib:nmorpaper}.  The magnetic shielding factors of each of
the four cylindrical shells, and of various combinations of them, were
measured and found to be consistent with $\mu_r\sim 20,000$.
%-40,000$ for
%3.5~$\mu$T applied external axial and transverse fields at frequencies
%$<0.1$~Hz, where the uncertainty in $\mu_r$ comes a model uncertainty
%resulting from incomplete inclusion of the geometry.

In our studies of the material properties of these magnetic shields,
two different approaches to measure $\mu(T)$ were pursued.  Both
approaches involved experiments done using witness cylinders made of
the same material and annealed at the same time as the prototype
magnetic shields.  We therefore expect they have the same magnetic
properties as the larger prototype shields, and they have the
advantage of being smaller and easier to perform measurements with.

The two techniques employed to determine $\mu(T)$ were the following:
\begin{enumerate}
\item measuring the low-frequency AC axial magnetic shielding factor
  of the witness cylinder as a function of temperature, and
\item measuring the temperature-dependence of the slope of a minor B-H
  loop, using the witness cylinder as a transformer core, similar to
  previous measurements of the temperature dependence of $\mu$, but
  for parameters closer to those encountered in nEDM experiments.
\end{enumerate}
We now discuss the details and results of each technique.

% Axial shielding factor measurements Section moved to axial.tex


\subsection{Axial Shielding Factor Measurements}

In these measurements, a witness cylinder was used as a magnetic
shield.  Small changes in the axial shielding factor are interpreted
as a change in the effective $\mu$ of the material.  This technique is
quite different than the usual mutual inductance techniques used by
other groups.  Our hope was that this geometry would give us more
confidence that the properties we were measuring were indeed related
to magnetic shielding properties rather than inherent properties of
the material.

\subsubsection{Magnetic Field Generation}

In order to measure the magnetic shielding factor, the witness
cylinder was placed within a homogeneous AC magnetic field.  The field
was created within the magnetically shielded volume of the prototype
magnetic shielding system (described previously in
Section~\ref{sec:XXX} in order to reduce unrelated magnetic noise
which might affect the measurement.  A solenoid inside the shielding
system was used to produce the magnetic field.  The radius and the
half length of the solenoid is 17.44~cm while the radius and the half
length of our innermost prototype passive shield is 18.44~cm.  In this
design, the field produced by the solenoid plus innermost shield
approximates an infinite solenoid producing a homogeneous magnetic
field internally.  The $\mu$ dependence of the reaction factor is
known to be considerably smaller than the $\mu$ dependence of the
axial magnetic shielding factor, hence any change in the fluxgate
amplitude with temperature can be ascribed to changes of the witness
cylinder, rather than changes in the innermost magnetic shield.

The solenoid has 14 turns with 2.6~cm spacing between the wires.  The
magnetic field generated by the solenoid was typically 1~$\mu$T.  The
solenoid current was varied sinusoidally at typically 0.01 to 10~Hz.

\subsubsection{Witness cylinder and fluxgate magnetometer}

The witness cylinder was placed into this magnetic field generation
system as shown schematically in Fig.~\ref{fig:geometry}.  The
cylinder was held in place by a wooden stand.

A Bartington fluxgate magnetometer Mag-03IEL70 (low noise) measured the
magnetic field at the center of the witness cylinder.


% while the
%temperature of the shield was measured.


To increase the resolution of the measured signal from the fluxgate, a
Bartington Signal Conditioning Unit (SCU) with a low-pass filter set
to typically 10-100~Hz and a gain set to typically $>50$ was used.
The signal from the SCU was demodulated by an SRS830 lock-in amplifier
providing the in-phase and out-of-phase components of the signal.
(The sinusoidal output of the lock-in amplifier reference output
itself was normally used to drive the solenoid generating the magnetic
field.)  The time constant on the lock-in was typically set to 3
seconds with 12~dB filter. The sensitivity settings was set by the
amplitude of the measured signal.
% describe the lock-in settings

\begin{figure}[h!]
\begin{center}
   \includegraphics[width=0.8\textwidth]{geometry.PNG}
    \caption{Axial shielding factor measurement setup. The witness
      cylinder with a radius of 2.54~cm and a length of 15.2~cm is placed
      inside a solenoid with a radius and a half length of
      17.44~cm. The axis of symmetry is along the $z$-axis. The
      windings of the solenoid is shown in red. The blue coil with one
      turn is coupled to the witness cylinder and it has a radius of
      5~cm.  }
% Note to Jeff (self): fix this figure caption!!!
% (Need to look at it in the pdf form to think more about it.)
    \label{fig:geometry}
    \end{center}
\end{figure}
% Add dimensions to figure.  Remove box and white space around
% diagram.


% To do (Taraneh and/or Jeff): where is the desciption of the
% temperature sensors!?!?!?  General data acquisition scheme?  How was
% the temperature even varied?  There are some serious missing pieces
% to the description of how the experiment was done!


% Note to Jeff: Done experiment description (for now - I expect some
% of the above is a part of what was discovered to be missing in
% writing the systematic errors section -- I will hope to fix it
% later).



\subsubsection{Data and Interpretation}

An example of the typical data acquired is shown in
Fig.~\ref{fig:B_vs_Temp}. Graph (a) shows the temperature of the
witness cylinder over four days. The temperature changes are about
3.5~K and are caused by temperature variations of the room. The
magnetic field $B$ is the sum, in quadrature, of the in-phase and
out-of-phase components. For $f\lesssim 1$ Hz most of the measured
fluxgate signal is in the in-phase component. The magnetic field
tracks the temperature trend in an opposite way which is shown in
graph (b). The slope of the graph (c) has been calculated by using a
linear fit to the data. In general, we measured 0.3 \%/K $< \vert
\frac{1}{B} \frac{dB}{dT} \vert <$ 2.3 \%/K with the sign being
negative. The range also encompasses the typical deviation from a
linear $B(T)$ in the data (for example, those seen in
Fig.~\ref{fig:B_vs_Temp}(c).), and so we adopt this as an
uncertainty. Through a comprehensive set of systematic studies, the
source of the deviations from linearity could not be found. We suspect
it is due to a combination of mechanical stability and properties of
the material that cannot be embodied by a single temperature
slope. Thus the experiment tends to set a scale and sign for the
possible temperature dependence, rather than a value.
  \begin{figure}[h!]
\begin{center}
   \includegraphics[width=0.5\textwidth]{B_vs_T.png}
    \caption{These graphs show data of temperature dependence of the
      magnetic field taken over 80 hours. Graph (a) shows the
      temperature of the witness cylinder in $^\circ$C as a function
      of time. Graph (b) shows the magnetic field inside the witness
      cylinder at the center in $\mu$T versus time and graph (c) shows
      the magnetic field in $\mu$T versus temperature in
      $^\circ$C. The red line is the linear fit to data. At
      22$^\circ$C, $\vert \frac{1}{B}\vert \vert
      \frac{dB}{dT}\vert\simeq -1.5 \% /K.$ }
    \label{fig:B_vs_Temp}
     \vspace{-2.em}
    \end{center}
\end{figure} 

To relate the data to $\mu(T)$, finite element simulations in FEMM and
OPERA were performed.  From these simulations the ratio $\frac{\mu}{B}
\frac{dB}{d\mu}$ was calculated in a model where $\bold{B}=\mu
\bold{H}$ and $\mu$ is a constant i.e. the material is treated as
being linear. The term $\frac{\mu}{B}\frac{dB}{d\mu}\neq 0$ because
the witness cylinders are open ended. Combining the measurement and
the simulations, the temperature dependence of effective $\mu$ (at
$\mu$=20000) can be calculated by
\begin{equation}
\frac{1}{\mu}\frac{d\mu}{dT}= -\frac{\frac{1}{B}\frac{dB}{dT}}{\frac{\mu}{B}\frac{dB}{d\mu}}.
\end{equation}

\subsubsection{Systematic Errors}

\paragraph{Initial Design}

%initial design stages
The initial design of the $\mu(T)$ measurement included Tygon tubing
wrapped around the witness cylinder in a spiral pattern to flow water
and change the temperature of the witness cylinder. In those
measurements, the amplitude of the signal was measured from a
Tektronix oscilloscope. The mechanical stability issues dominated the
systematic uncertainties. When water was running, the flexibility of
the tubing caused a movement in the witness cylinder. As an
alternative, the tubes were replaced by copper tubing. In this case,
the challenge was to create enough contact between the tubes and the
witness cylinder which was not successful. In another design, a TEC
was replaced with the tubing. The main issue with this design was that
it did not provide enough cooling for the witness cylinder and also it
was creating only local temperature changes on the witness
cylinder. In addition, despite of using heat sinks, the heat created
by the TEC itself made it very inefficient. To reduce the background
noise in this measurement, all the components interior to the
measurement volume should be non-magnetic. As a result, type T
thermocouples were used which has copper and constantan conductors.


% T-type are less magnetic than K-type.  How much less and what's the
% systematic error, since they are not nonmagnetic?  I would say it is
% unknown.  Is there a way to assign a number to this?
%Taraneh: I couldn't find any number related to this. I am not sure if the fluxgates are sensetive enough to measure it too.

\paragraph{Final Design} 

%final design
The final design included the witness cylinder, which was placed with
a holder inside the innermost prototype passive shield. The
temperature variations of the witness cylinder was due to the
temperature changes in the room which was following the on and off
cycles of the building air conditioner. A Bartington magnetic field
sensor (fluxgate) was hold inside the witness cylinder at the centre
by plastic holder. The signal from the sensor was then demodulated by
an SRS830 lock-in amplifier. To convert the final measured voltage
from the sensor to $\mu$T, the sum in quadratures of the in-phase and
out of phase components of the lock-in amplifier was multiplied by the
scaling factor on the magnetic field sensor which is 7 $\mu$T/V.  The
mu-metal alloy has a thermal conductivity of 0.35
W/(cm$\cdot$K). Therefore, four thermocouples placed on different
spots on the witness cylinder to correct for small temperature
differences on each end of the witness cylinder.  In these
measurements, the farther side of the passive shields had the end caps
while the other side was left open to give access to the interior
region. Based on the magnetic field maps inside the passive shields,
for 10 $\mu$T applied magnetic field, the stability of the field was
to 0.1 $\mu$T level and the magnetic field was even more stable closer
to the closed side of the passive shields. Therefore the witness
cylinder was pushed to the side with the end caps on the passive
shields.

% Another list
% - amplitude measurement on oscilloscope (no lock-in). (done)
% - speed of temperature change? temperature homogeneity?  (not here yet) 
% - other methods of cooling and why they didn't work. (not here yet)(done)
% - magnetic contamination by thermocouples (done)

% The above list maybe explains other possible variations on the
% method that you tried that all seemed to have worse systematic
% errors.

% Then, a list of systematic errors once you settled on the final
% measurement technique...

% - field profile/homogeneity (done)
% - mechanical stability (done)
% - thermal expansion (done)
% - temperature dependence of fluxgate and SCU (done)
% - temperature dependence of lock-in (done)
% - temperature dependence of coil resistance (not here yet) (done)
%    - Cupron/etc. study. plus
%    - measuring across the precision resistor
% - null measurement (Cu, and why it doesn't give the whole story, not here yet) (done)
% - effect of endcaps? (influence of external fields, not here yet)(done)
% - temperature dependence of reaction factor (negligible, not here yet)(done)
% - degaussing (not here yet)(done)
% - measurements on different witness cylinders? (not here yet)(done)
% - vibrations of the building?(done)
% - ...
%
% Another list is on the oldelog somewhere.  Please find it.
% The above list is longer and inclusive of the old list.
% See oldelog entry 141 in Magnetic Fields.
%
% This list of systematics is addressed now...


To eliminate the coupling of the field profile and mechanical
stabilities, another coil was used which was coupled to the witness
cylinder as shown in Fig.~\ref{fig:geometry}.  This also enabled us to
study the dependence of the field profile to $\mu(T)$. The result of
this study did not show a significant change and data was similar to
Fig. \ref{fig:B_vs_Temp} (c). Therefore, the systematic error due to
the field profile is smaller than the unknown systematic errors of
Fig. \ref{fig:B_vs_Temp} and similar figures.
% What is the value of the systematic error deduced from this study?
% I would say that the result of this study was that we got similar
% data to Fig. 3(c).  Therefore the systematic error due to field
% profile is smaller than the unknown systematic error of Fig. 3(c)
% and similar figures.  also partly addresses possible changes in
% reaction factor with temperature. (done)


%Temperature dependence of fluxgate and SCU
In these measurements, a Mag-03IEL70 Bartington magnetic field sensor which has individual sensor elements and a Mag-03SCU signal conditioning unit were used.
The magnetic field sensor is a low noise sensor with a range of $\pm$70 $\mu$T and it has a scaling temperature coefficient of 15 ppm/K.


As another mechanical stability study, the movement of the Bartington
fluxgate flying lead due to thermal expansion was estimated. If the
fluxgate flying lead move about 1 mm normal to its axis of symmetry
which is parallel to the axis of the witness cylinder, the magnetic
field will change about 30~ppm/K over 20~K temperature changes.  The
thermal expansion of the mu-metal cylinder is at the order of
10~ppm/K~\cite{kruppvdm}.  An SRS830 lock-in amplifier has 50
ppm/K amplitude stability. It is documented in the operating
manual \cite{bib:lockin}.

%We did this for the transformer technique
The temperature dependence of the coil resistance was also measured by
measuring the voltage across a temperature controlled 1~$\Omega$
resistor.
To address this effect, another measurement conducted with local coil with Cupron wire windings. Cupron has higher resistivity compared to copper wires. The results showed no linear correlation between the measured magnetic field and the temperature of the witness cylinder.
%The results?

%Taraneh: I calculated from the long summary
The stability of the system was tested by replacing the mu-metal
witness cylinder with a similar copper cylinder. For all those
measurements the stability was $<0.1$\%/K. Although these type of
measurements help to find an error bar on the stability of the system
in general, they do not help to find the sources of the systematic
uncertainty.


%effect of the end caps
In this experiment, one side of our prototype passive shield was closed with the end caps while the other side left open for easier access to the interior region. The result of the magnetic field map inside the prototype shield when the solenoid was  turned on showed that closer to the far side of the shield, where the shield is closed, the magnetic field is more uniform. For a 10$\mu$T applied magnetic field the maximum change of the magnetic field along the axis of the passive shield was the order of 0.1~$\mu$T.
Therefore the witness cylinder was pushed to the far side of the prototype shield to reduce the non uniformity of the magnetic field.
%This is from oldElog entry 146: field map by Andrew Harrison

%temperature dependence of reaction factor
The changes in magnetic permeability has an effect on the measured magnetic field inside the witness cylinder. Since the reaction factor for this geometry is not strongly correlated to $/mu$ in Fig. \ref{fig:Magnetic_Field}, the changes of the magnetic field due to magnetic permeability changes in negligible. For XXX change in $\mu$ the magnetic field internal to the witness cylinder changes by XXX. 

%degaussing
The magnetization of the prototype passive shield changes the magnetic permeability of the material and so the reaction factor changes. The degaussing procedure has a considerable effect on the initial magnetic field.
%what next?

%different witness cylinders
Three witness cylinders were tested for these measurements. In most cases, despite keeping the conditions the same, the results were significantly different for each witness cylinder. The systematic studies could not address these sources of error. As a result these changes is believed to be driven by the intrinsic properties of the witness cylinders.   
In general, annealing process and take-out temperatures have a strong effect on the properties of the shields and so their $\mu$ value.
%vibration
Since the experiment site is located at the heart of downtown it is also possible that vibrations of the building affected the experiment setup and its machanical stability.

Although for most of the measurements the general trend of $B(T)$ graphs was consistent, the shape and positions of the nonlinear parts of $B-T$ graphs were changing.


% Two possibilities:

% 1. unknown systematic error(s)

% 2. complicated material properties that don't follow a straight line
% could depend on material history themselves.

% Therefore we state a range which should be indicative of the scale
% of possible temperature dependence.


Taking all these systematics into account, in this method it was found that 0.6\%/K
$\lesssim\frac{1}{\mu}\frac{d\mu}{dT}\lesssim 2.3\%$/K. 



%\begin{itemize}
%\item Describe experimental setup and important considerations
%  (e.g. relationship of data to effective $\mu$)
%\item Explain B, H, f, and dominant systematic effects.
%\item One figure of experimental setup?
%\item One data graph?
%\item State overall result and systematic error.
%\end{itemize}


% Transformer core measurements Section moved to transformer.tex


\subsection{Transformer Core Measurements}
In second approach, the witness cylinder was used as the core of a
transformer. Two coils (primary and sense coils) were wound on the
witness cylinder using copper wire. For this technique, three witness cylinder used and each had different number of winding on primary and secondary coil. This enabled more saturation of the material. The primary coil generated an AC magnetic field
using the internal oscillator of a SRS830 lock-in amplifier while the
sense coil measured the sense voltage and picked up the magnetic
field flux produced by the primary coil. This experiment was done at 1
Hz and as small as possible $H$, the order of 0.1~A/m,
% state typical value
to measure the slope of the minor $B-H$ loops near the origin of the
$B-H$ space. In lower frequencies, noise dominated the actual signal.
% why?

The temperature of the core was measured at the same time as the sense voltage by using non-magnetic type T thermocouples. The measured voltage induced on the sense coil, is proportional to $\mu$.  
%The measurements show a range of 0.1\%/K to 2\%/K for $\frac{1}{\mu}\frac{d\mu}{dT}$, again assuming the material to be linear.

In all transformer measurements, the sense voltage from the sense coil
was demodulated by a lock-in amplifier. In
Fig.~\ref{fig:data_and_simulation} the in-phase component $Y$ and
out-of-phase component $X$ of the sense signal as a function of
applied $H$ field at 1 Hz are shown. In the ideal case where the
material is linear, the $X$ component should be zero. But our observation showed a nonzero value for the $X$ component at all frequencies at a typical applied $H$ field. At higher frequencies the value of $X$ dominates the measured signal.
Therefore, the ratio of $X/Y$ depends on both the applied frequency and the position on the
$B-H$ region. We were interested in regions where $\vert Y \vert >
\vert X \vert $ in which the material behaves as being linear and the
$\vert H \vert $ values correspond to $\vert X \vert < \vert X_{max}
\vert$ while still being able to get a good signal to noise ratio.


%A concern in both measurements is that, the material is not linear. 
The observation of nonzero $X$ signal is due to effects such as the nonlinearity of the witness cylinders, such as hysteresis, saturation, eddy currents and skin depth. Therefore, it is not possible to embody the results as a single parameter.
To check the sensitivity of the sense coil demodulated voltage to
these effects, a theoretical model of the hysteresis (Jiles-Atherton
model) was used \cite{bib:jiles}, see Fig.
\ref{fig:data_and_simulation}.

The key parameters of Jiles-Atherton model are saturation
magnetization $M_s$, mean field parameter $\alpha$ which represents
interdomain coupling, $a$ with the dimensions of magnetic field which
characterizes the shape of anhysteretic magnetization, bulk
magnetization $M$ and the magnetic field $H$. A shown in
Fig. \ref{fig:data_and_simulation}, trends in the measurements and
simulations are fairly consistent.


\begin{figure}[h!]
\begin{center}
   \includegraphics[width=0.5\textwidth]{data_and_simulation3.PNG}
    \caption{These graphs represent the sense voltage as a function of applied $H$ field at 1 Hz. Graph (a) shows the out-of-phase component $X$ of the measured sense voltage as well as the simulation. Graph (b) shows the in-phase component $Y$ of the signal and its simulation.}
    \label{fig:data_and_simulation}
    \end{center}
\end{figure} 

\subsubsection{Systematic Errors}

% Need list
% Please place here
%
% 1.  Focus on unique ones:
% - motion of wires (done)
% - degaussing, how done, and general observations (done)
% - resistance of wires (done)
% - different numbers of turns on primary/secondary  (Taraneh: talked about it earlier)
% - in-phase/out-of-phase components, how small should in-phase be? (Taraneh: talked about it)
% - background noise (done)
% - ...

% 2.  Or if there are any other ones that are from the same source
% (e.g. resistance of wire might be in this category) but the value is
% different for some reason

% 3.  Or if some are eliminated, it would be good to note here, for
% example, fluxgate systematics and some classes of motion systematics


The dominant source of systematic uncertainty in this method arises
from the properties of witness cylinders. 

%motion of wires
Since the coils were wound by hand the spacing between each loop of wire was not precise and it might lead to the motion of the wires during data acquisition. Although it might effect the measurement results, there was no way to address this systematic effect.


%degaussing
Some measurements conducted to study the effect of magnetization on the measured sense voltage. Degaussing
(idealizing) magnetic shields means to saturate the shields by
applying a very large magnetic field and then slowly turn the
applied field down to zero. In our case it was done by a C code controlling the lock-in amplifier. A sine wave with a known frequency, typically 1~Hz was applied on the primary coil. The minimum voltage change in a lock-in amplifier is 2~mV. The applied voltage was changing from 5~V to 4~mV which is the minimum possible voltage on a lock-in amplifier. It was also possible to adjust the speed of the degaussing as a time delay was implemented in the code.
The voltage was then increased to reach the typical $H$ field.
A comparison between the $X$ and $Y$ components of the sense voltage before and after degaussing was then made to study the effect.
In most cases, degaussing helped to
increase both the $X$ and $Y$ signal and hence $\mu$ although it did not change its temperature dependence significantly. Very long degaussing has a reverse effect on the measured signal.
% But it didn't change its temperature dependence significantly.  Poor
% degaussing could result in initial slopes that were large for a
% while?


%resistance of wire
To check the stability of the applied current on the primary coil, voltage across a 1~$\Omega$ temperature controlled resistor in series with the primary coil was measured. The result showed about 100~ppm/K changes in current.
%Taraneh: does this number make sense? I might need to double check it again.

%background noise
To reduce the background noise and
thermally isolate the witness cylinder, some measurements were done by
placing the witness cylinder inside the prototype passive shield. The
results did not show a significant change.


Saturation and magnetization of the witness cylinders and the nonlinearity of $/mu$ in general had
a considerable effect on the measured signal. Despite of the
consistent general trend of $B(T)$, different witness cylinders tend
to behave differently under similar conditions. 












%\begin{itemize}
%\item Describe experimental setup and important consdierations (e.g. relationship of data to effective $%\mu$)
%\item Explain B, H, f, and dominant systematic effects.
%\item Understanding in terms of Jiles-Atherton.  Complications that
 % this does not translate well into ``$\mu$''.
%\item One data graph?  (Perhaps the Jiles-Atherton one?)
%\item State overall result and systematic error.
%\end{itemize}
%%%%%%%%%%%%%%%%%%%%%%%%%%%%%%%%%%%%%%%%%%%%%%%%%%%%%%%%%%%%%%%%%%%%%%
% Taraneh, can you write text up to this point?
%%%%%%%%%%%%%%%%%%%%%%%%%%%%%%%%%%%%%%%%%%%%%%%%%%%%%%%%%%%%%%%%%%%%%%


\section{Relationship to nEDM experiments\label{sec:relationship}}
%\subsection{Summary of $\mu(T)$}

Neutron EDM experiments are typically designed with the DC coil being
magnetically coupled to the innermost magnetic shield.  As discussed
in Section~\ref{sec:calculation}, if the magnetic permeability of the
shield changes, this results in a change in the field in the
measurement region by an amount
$\frac{\mu}{B_0}\frac{dB_0}{d\mu}=0.01$.

The temperature dependence of $\mu$ has been constrained by two
different techniques using open-ended mu-metal witness cylinders
annealed at the same time as our prototype magnetic shields.  We
summarize the overall result as
0.0\%/K~$<\frac{1}{\mu}\frac{d\mu}{dT}<$~2.7\%/K, where the range is
driven in part by material properties of the different mu-metal
cylinders, and in part by day-to-day fluctuations in the temperature
slopes.



We note the following caveats in relating this measurement to nEDM
experiments:
\begin{itemize}
\item Although the measurement techniques rely on considerably larger
  frequencies and different $H_m$-fields than those relevant to
  typical nEDM experiments, we think it reasonable to assume the
  temperature dependence of the effective permeability should be of
  similar scale.  For frequency, both techniques typically used a 1~Hz
  AC field, whereas for nEDM experiments the field is DC and stable at
  the 0.01~Hz level.  Furthermore, in one measurement technique the
  amplitude of $H_m$ was $\sim 0.004$~A/m and in the other was $\sim
  0.1$~A/m.  For nEDM experiments $H_m<0.007$~A/m and is DC.
\item Both measurement techniques extract an effective $\mu$ that
  describes the slope of minor loops in $B_m-H_m$ space.  A more
  correct treatment would include a more comprehensive accounting of
  hysteresis in the material, which is beyond the scope of this work.
\end{itemize}

Assuming our measurement of
0.0\%/K~$<\frac{1}{\mu}\frac{d\mu}{dT}<2.7$\%/K and the generic EDM
experiment sensitivity of $\frac{\mu}{B_0}\frac{dB_0}{d\mu}=0.01$
results in a temperature dependence of the magnetic field in a typical
nEDM experiment of $\frac{dB_0}{dT}=0-270$~pT/K.  To achieve a goal of
$\sim 1$~pT stability in the internal field for nEDM experiments, the
temperature of the innermost magnetic shield in the nEDM experiment
should then be controlled to the $<0.004$~K level if the worst-case
dependence is to be taken into account.  This represents a potentially
challenging design constraint for future nEDM experiments.

As noted by others~\cite{bib:cpviolwithoutstrangeness}, the use of
self-shielded coils to reduce the coupling of the $B_0$ coil to the
innermost magnetic shield is an attractive option for EDM experiments.
The principle of this technique is to have a second coil structure
between the inner coil and the shield, such that the net magnetic
field generated by the two coils is uniform internally but greatly
reduced externally.  For a perfect self-shielded coil, the field at
the position of the magnetic shield would be zero, resulting in
perfect decoupling, which is to say a reaction factor that is
identically unity.  For ideal geometries, such as spherical
coils~\cite{bib:brown, bib:wheeler,bib:purcell} or infinitely long
sine-phi coils~\cite{bib:bethBNL,bib:bethUSpatent,bib:bidinosti}, the
functional form of the inner and outer current distributions are the
same, albeit with appropriately scaled magnitudes and opposite sign.
More sophisticated analytical and numerical methods have been used
extensively in NMR and MRI to design self-shielded
gradient~\cite{bib:turner,bib:hidalgo},
shim~\cite{bib:brideson,bib:forbes}, and transmit
coils~\cite{bib:bidinosti,bib:kuzmin}, and should be of value in the
context of nEDM experiments, as well.  We are also pursuing novel
techniques for the design of self-shielded coils of any arbitrary
field profile and geometric shape~\cite{bib:crawford}.



\section{Conclusion}

In the axial shielding factor measurement, we found
0.6\%/K~$<\frac{1}{\mu}\frac{d\mu}{dT}<2.7\%$/K, with the measurement
being conducted with a typical $H_m$-amplitude of 0.004~A/m and at a
frequency of 1~Hz.  In the transformer core case, we found
0.0\%/K~$<\frac{1}{\mu}\frac{d\mu}{dT}<2.2\%$/K, with the measurement
being conducted with a typical $H_m$-amplitude of 0.1~A/m and at a
frequency of 1~Hz.

The primary caveat to these measurements is that both measurements
(transformer core and axial shielding factor) do not truly measure
$\mu$.  Rather they measure observables related to the slope of minor
hysteresis loops in $B_m-H_m$ space.  They would be more appropriately
described by a hysteresis model like that of Jiles~\cite{bib:jiles},
but to extract the temperature dependence of all the parameters of the
model is beyond the scope of this work.  Instead we acknowledge this
fact and relate the temperature dependence of the effective $\mu$
measured by each experiment.

We think it is interesting and useful information that the two
experiments measure the same scale and sign of the temperature
dependence of their respective effective $\mu$'s.  This is a principal
contribution of this work.

In future work, we plan to measure $B_0(T)$ directly for nEDM-like
geometries using precision atomic magnetometers.  We anticipate based
on the present work that self-shielded coil geometries will achieve
the best time and temperature stability.



\section{Acknowledgments}

We thank D.~Ostapchuk from The University of Winnipeg for technical
support.  We gratefully acknowledge the support of the Natural
Sciences and Engineering Research Council Canada, the Canada
Foundation for Innovation, and the Canada Research Chairs program.


\section*{References}

\bibliography{mybibfile}
\begin{thebibliography}{00}

\bibitem{bib:nedm2} A. P. Serebrov {\it et al.}, JETP Lett. {\bf 99}, 4
  (2014).

\bibitem{bib:nedm2.5} A. P. Serebrov {\it et al.}, Phys. Procedia {\bf
  17}, 251 (2011).

\bibitem{bib:nedm3} K. Kirch, AIP Conf. Proc. {\bf 1560}, 90 (2013).

\bibitem{bib:nedm3.5} C. A. Baker, {\it et al.}, Phys. Procedia {\bf
  17}, 159 (2011).

\bibitem{bib:nedm5} I. Altarev, {\it et al.}, Nuovo Cim. C {\bf
  35}, 122 (2012).

\bibitem{bib:nedm6} R. Golub and S. K. Lamoreaux, Phys. Rept.  {\bf
  237}, 1 (1994).

\bibitem{bib:nedm6.5} T. M. Ito (for the nEDM Collaboration),
  J. Phys. Conf. Ser. {\bf 69} 012037, 2007.

\bibitem{bib:nedmtriumf} R.~Picker (for the TRIUMF Japan-Canada UCN
  Collaboration), in the proceedings of MENU2016, July 25-30, 2016,
  Kyoto, Japan, arXiv:1612.00875 [physics.ins-det].

\bibitem{bib:baker} C. A. Baker, {\it et al.}, Phys. Rev. Lett. {\bf
  97}, 131801 (2006).

\bibitem{bib:pendlebury} J. M. Pendlebury {\it et al.}, Phys. Rev. D
  {\bf 92}, 092003 (2015).

\bibitem{bib:brys} T. Bry\'s, {\it et al.}, Nucl. Instrum. Meth. A
  {\bf 554}, 527 (2005).

\bibitem{bib:afach} S. Afach, {\it et al.}, J. Appl. Phys. {\bf 116},
  084510 (2014).

\bibitem{bib:fierlingerroom} I. Altarev, {\it et al.}
  Rev. Sci. Instrum. {\bf 85}, 075106 (2014).

\bibitem{bib:sturmthesis} M. Sturm, Masterarbeit, T.U. M\"unchen
  (2013).

\bibitem{bib:patton} B. Patton, E. Zhivun, D. C. Hovde, and D. Budker,
  Phys. Rev. Lett. {\bf 113}, 013001 (2014).


\bibitem{bib:altarev2014} I. Altarev {\it et al.},
  Rev. Sci. Instrum. {\bf 85}, 075106 (2014).


\bibitem{bib:altarev2015} I. Altarev {\it et al.}, J. Appl. Phys. {\bf 117}, 233903 (2015).



\bibitem{bib:voigt} J. Voigt {\it et al.}, Metrol. Meas. Syst. {\bf 20}, 239 (2013).

\bibitem{bib:thiel} F. Thiel {\it et al.}, Rev. Sci. Instrum. {\bf 78}, 035106 (2007).

\bibitem{bib:fierlinger2016} Z. Sun {\it et al.}, J. Appl. Phys. {\bf
  119}, 193902 (2016).

\bibitem{bib:franke} B. Franke, PhD Thesis, ETH Z\"urich (2013).

\bibitem{bib:couderchon} G. Couderchon, J. F. Tiers,
  J. Magn. Magn. Mat. {\bf 26}, 196 (1982).

\bibitem{bib:kruppvdm} Krupp VDM Magnifer 7904, Material Data Sheet
  No.~9004, Aug. 2000, Krupp VDM GmbH, Postfach 18 20, D-58778
  Werdohl, Germany.

\bibitem{bib:gupta} K. Gupta, K. K. Raina, S. K. Sinha, J. Alloys
  Compd. {\bf 429}, 357 (2007).

\bibitem{bib:bozorth} R. M. Bozorth, {\it Ferromagnetism} (IEEE Press,
  Piscataway, NJ, 1993).

\bibitem{bib:bidinostimartin} C. P. Bidinosti, J. W. Martin, AIP Advances
  {\bf 4}, 047135 (2014).

\bibitem{bib:urankar} L. Urankar, R. Oppelt, IEEE
  Trans. Biomed. Eng. {\bf 43}, 697 (1996).

\bibitem{bib:knecht} A. Knecht, PhD Thesis, U. Z\"urich (2009).

\bibitem{bib:femm} Finite Element Method Magnetics FEMM version 4.2,
  available from {\tt http://www.femm.info}.


\bibitem{bib:lambert} R.H.\ Lambert and C.\ Uphoff, Rev. Sci. Instrum. {\bf 46},  337 (1975).

\bibitem{bib:sumner} T.J.\ Sumner,  J. Phys. D: Appl. Phys. {\bf 20} 692 (1987).




%\bibitem{bib:smythe} W. R. Smythe, {\it Static and Dynamic
%  Electricity} (McGraw-Hill, New York, 1950).

%\bibitem{bib:ferraro} V. C. A. Ferraro, {\it Electromagnetic Theory}
%(Athalone Press, London, 1967).

\bibitem{bib:pfeifer} F. Pfeifer and C. Radeloff,
  J. Magn. Magn. Mat. {\bf 19}, 190 (1980).

\bibitem{bib:nmorpaper} J. W. Martin {\it et al.},
  Nucl. Instrum. Meth. A {\bf 778}, 61 (2015).

\bibitem{bib:bartman} Bartington Instruments Ltd., 10 Thorney Leys
  Business Park, Witney, Oxon, OX28 4GG, England.

\bibitem{bib:lockin} Stanford Research Systems, 1290-D Reamwood Ave.,
  Sunnyvale, CA 94089.

\bibitem{bib:paperno-open-ended} E. Paperno, IEEE Trans. Magn. {\bf
  35}, 3940 (1999).

\bibitem{bib:jiles} D. C. Jiles, J. Appl. Phys. {\bf 76}, 5849 (1994).

\bibitem{bib:jiles-atherton} D. C. Jiles and D. L. Atherton,
  J. Appl. Phys. {\bf 55}, 2115 (1984); D. C. Jiles and D. L. Atherton,
  J. Magn. Magn. Mat. {\bf 61}, 48 (1986).



\bibitem{bib:cpviolwithoutstrangeness} I. B. Khriplovich,
  S. Lamoreaux, {\it CP violation without strangeness: electric dipole
    moments of particles, atoms, and molecules.} (Springer-Verlag,
  Berlin, 2012).



\bibitem{bib:brown} W. F. Brown Jr. and J. H. Sweer, Rev. Sci. Instrum. {\bf 16},  276 (1945).

\bibitem{bib:wheeler} H. A. Wheeler, Proceedings of the IRE {\bf 46},1595 (1958).  %DOI: 10.1109/JRPROC.1958.286978

\bibitem{bib:purcell}E.M.\ Purcell, %Helmholtz coils revisited, 
Am. J. Phys. \textbf{57}, 18 (1989); Am. J. Phys. \textbf{58}, 296 (1990). %http://dx.doi.org/10.1119/1.15860 and http://dx.doi.org/10.1119/1.16156

%\bibitem{bib:storey} J. R. Storey, Magnetic sensors with good signal-to-interference discrimination, Il Nuovo Cimento D, \textbf{2}, 153 (1983). % doi:10.1007/BF02455919



\bibitem{bib:bethBNL} R.A.\  Beth, Brookhaven National Laboratory Report %Combination of cylindrical current sheets to cancel external field,  
BNL-10143 (1966). % DOI: 10.2172/4513223.

\bibitem{bib:bethUSpatent} R.A.\  Beth, %Cancellation of external magnetic fields by inner and outer cylindrical current sheets, 
US Patent 3466499, September 9, 1969.


\bibitem{bib:bidinosti} C.P. Bidinosti, I.S. Kravchuk, and
  M.E. Hayden, J. Magn. Res. {\bf 177}, 31 (2005).
  

\bibitem{bib:turner} R.\ Turner and R.M.\ Bowley, %Passive screening of switched magnetic field gradients, 
J. Phys. E: Sci. Instrum. \textbf{19}, 876 (1986). % http://dx.doi.org/10.1088/0022-3735/19/10/023

\bibitem{bib:hidalgo} S.S.\ Hidalgo-Tobon, %Theory of gradient coil design methods for magnetic resonance imaging, 
Concepts Magn. Reson. \textbf{36A}, 223 (2010).  % DOI: 10.1002/cmr.a.20163

\bibitem{bib:brideson} M.A. Brideson, L.K. Forbes, S. Crozier, Concepts Magn. Reson. {\bf 14}, 9 (2002).

\bibitem{bib:forbes} L.K. Forbes and S. Crozier, J. Phys. D: Appl. Phys {\bf 36}, 68 (2003).

\bibitem{bib:kuzmin} V.V.\ Kuzmin  {\it et al.}, J. Magn. Reson. {\bf 256}, 70 (2015).

\bibitem{bib:crawford} C. Crawford, \textit{private communication}.





\end{thebibliography}

\end{document}
