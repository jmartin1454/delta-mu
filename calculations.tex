

\section{Sensitivity of Internally Generated Field to Permeability of the Shield $B_0(\mu)$\label{sec:calculation}}

The presence of a coil inside the innermost passive shield turns
 the shield into a return yoke, and generally results in an incease
 in the magnitude of $B_0$. %of the central magnetic field $B_0 (0,0,0)$. 
 The ratio of this field inside the 
coil in the presence of the magnetic shield to that of the coil in
free space is referred to as the reaction factor $C$, and can be calculated analytically
for spherical and infinite cylindrical geometries~\cite{bib:bidinostimartin,bib:urankar}. 
The key issue of interest for this work is the dependence of the
reaction factor on the permeability $\mu$ of the innermost shield.    Although this
dependence can be rather weak, the constraints on $B_0$ stability are very stringent.
As a result, even a small change in the magnetic properties of the innermost shield
can result in an unacceptably large change in $B_0$.


To illustrate, we consider here the model of a sine-theta surface current on a sphere of radius $a$, inside a spherical shell of inner radius $R$, thickness $t$, and linear permeability $\mu$.  
%
The uniform internal field generated by this ideal spherical coil is augmented by a factor $C$ in the presence of the shield, but is otherwise left undistorted.  The general  reaction factor for this model is given by Eq.~(38) in Ref.~\cite{bib:bidinostimartin}.  In the high-$\mu$ limit, with $t\ll R$,  the reaction factor can be approximated as
%
\begin{equation}
C 
 \simeq 1+ \frac{1}{2}\, \left( \frac{a}{R} \right)^{3} \left( 1- \frac{3}{2} \, \frac{R}{t} \, \frac{\muo}{\mu} \right) \, ,
 \label{Csphere}
\end{equation}
which highlights the dependence of $B_0$ on the relative permeability $\mu_r=\mu/\muo$ of the shield. 



Figure~\ref{fig:Magnetic_Field} shows plots of $B_0$ versus $\mu_r$ for coil and shield dimensions
similar to the ILL nEDM experiment~\cite{bib:baker,bib:knecht}:  $a=0.53$~m, $R= 0.57$~m, and  $t=1.5$~mm.
In addition to analytic calculations, we also include the results of
two axially symmetric simulations conducted using FEMM~\cite{bib:femm}
 %finite-element simulations that were conducted
  to assess the effects of geometry and discretization of the surface current.

%The dashed curve represents the
%results of an analytical calculation for a perfect spherical surface
%current.  For this calculation, a coil of radius 0.53~m inside a
%magnetic shield with inner radius $0.57$~m and thickness 1.5~mm were
%used.  The dimensions have been selected to be comparable to the
%dimensions of the ILL nEDM experiment
%geometry~\cite{bib:baker,bib:knecht}.

\begin{figure}[h!]
\begin{center}
   \includegraphics[width=0.7\textwidth]{femm_and_calcs_CB.pdf}
    \caption{Magnetic field at the coil isocenter as a function of shield
      permeability  for a geometry similar to the ILL nEDM
      experiment as discussed in the text.  The solid curve is the exact calculation for the ideal spherical coil and shield from Ref.~\cite{bib:bidinostimartin}; the dashed curve is the approximation of Eq.~\ref{Csphere}. The  circles and squares are the
      FEMM-based simulations for the spherical and solenoidal geometries
      with discrete currents.  In all cases, current magnitudes  were chosen to give $B_0=1~\mu$T at 
      $\mu_r=20,000$.}
    \label{fig:Magnetic_Field}
    \end{center}
\end{figure} 


%The calculations are compared with finite-element analysis simulations
%which were conducted to analyze the effect of discretizing the surface
%current, and to test the geometry dependence of the sensitivity of the
%experiment to changes in $\mu$.  

%Two axially symmetric simulations were conducted using FEMM~\cite{bib:femm}.  

In the first simulation,
the same spherical geometry was used as for the analytical
calculations.  However, the surface current was discretized to 50
individual current loops, inscribed onto a sphere, and equally spaced
vertically (i.e.~a discrete sine-theta coil).  A square wire profile
of side length 1~mm was used.  As shown in
Fig.~\ref{fig:Magnetic_Field}, this simulation gave good agreement
with the analytical calculation.
%
In the second simulation, a solenoid  and cylindrical shield (length/radius = 2) were used with the same dimensions as above.
Similarly, the coil was modelled as 50 evenly spaced current loops, 
with the distance from an end loop to the inner face of the
shield end-cap being half the inter-loop spacing.  In the limit of tight-packing 
(i.e., a continuous surface current) and  infinite $\mu$, the image currents in the end caps of
the shield act as an infinite series of current loops, giving the ideal uniform field of an 
infinitely  long
solenoid~\cite{bib:lambert,bib:sumner}.
As shown in Fig.~\ref{fig:Magnetic_Field}, the result is similar to the spherical case, though the slope of $B_0(\mu_r)$ is
somewhat steeper.

In ancillary measurements of shielding factors (discussed briefly in
Section~\ref{sec:previousmeasurement}), we found $\mu_r=20,000$ to
offer a reasonable description of the quasistatic shielding factor. By
evaluating the slope of the analytic curve in Fig.~\ref{fig:Magnetic_Field} at
$\mu_r=20,000$, we estimate that the scale of the sensitivity of a
generic nEDM experiment to global changes in the magnetic permeability
is $\frac{\mu}{B_0}\frac{dB_0}{d\mu} \sim \frac{3 a^3 }{4R^2 t} \frac{1}{\mu_r} \sim 0.01$.

%In the spherical case the reaction factor is 1.39, meaning that the
%field generated by the spherical coil is amplified by this
%multiplicative factor by the presence of the magnetic shield (flux
%return).  In the cylindrical case the reaction factor is 1.35.  If the
%reaction factor is closer to unity, the calculated
%$\frac{\mu}{B_0}\frac{dB_0}{d\mu}$ will be smaller, and the EDM
%experiment less sensitive to changes in $\mu$.  The results in the two
%geometries show that there is also geometry dependence.


For a high-$\mu$ innermost shield, the magnetic field lines emanating
from the coil all return through the shield.  This principle can be
used to estimate the magnetic field internal to the material $B_m$,
and in our studies gave good agreement with FEA-based simulations.
For the solenoidal geometry previously described and used for the
calculations in Fig.~\ref{fig:Magnetic_Field}, $B_m$ is largest in the
side walls of the solenoidal flux return, attaining a maximum value of
170~$\mu$T.  If we assume $\mu_r$=20,000, the $H_m$ field is
0.007~A/m.  Typically the shield is degaussed (idealized) with the
internal coil energized.  After degaussing, $B_m$ must be
approximately the same, since all flux returns through the shield.
However, the $H_m$ field must become significantly smaller, as the
material must reside on the ideal magnetization curve in $B_m-H_m$
space.  (For a discussion of the ideal magnetization curve, we refer
the reader to Ref.~\cite{bib:bozorth}.)  In principle, the $H_m$ field
could be reduced by an order of magnitude or more, depending on the
steepness of the ideal magnetization curve near the origin.  Thus
$B_m=170~\mu$T and $H_m<0.007$~A/m set a scale for the relevant values
for nEDM experiments.  Furthermore, the field in the nEDM measurement
volume, as well as in the magnetic shield, must be stable for periods
of typically hundreds of seconds (corresponding to frequencies
$<0.01$~Hz).  This sets the relevant timescale for magnetic properties
most relevant to nEDM experiments.
