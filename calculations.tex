\section{Sensitivity of Internally Generated Field to Permeability of the Shield $B_0(\mu)$\label{sec:calculation}}

The presence of a coil inside the innermost passive shield turns the
shield into a return yoke.  The ratio of the magnetic field inside the
coil in the presence of the magnetic shield to that of the coil in
free space is called the reaction factor.  Normally the reaction
factor is larger than unity for spherical and cylindrical geometries.
The key issue of interest for this work is the dependence of the
reaction factor on $\mu$.  This factor can be calculated analytically
for spherical and infinite cylindrical
geometries~\cite{bib:bidinostimartin,bib:urankar}.  Although the
dependence of the reaction factor on $\mu$ is rather weak for these
geometries, the constraints on $B_0$ stability are very stringent.
Thus a small change in the magnetic properties of the innermost shield
can result in an unacceptably large change in $B_0$.

Fig.~\ref{fig:Magnetic_Field} shows the central magnetic field $B_0$
as a function of relative $\mu_r=\mu/\mu_0$ for a geometrical size
typical of an nEDM experiment.  The dashed curve represents the
results of an analytical calculation for a perfect spherical surface
current.  For this calculation, a coil of radius 0.53~m inside a
magnetic shield with inner radius $0.57$~m and thickness 1.5~mm were
used.  The dimensions have been selected to be comparable to the
dimensions of the ILL nEDM experiment
geometry~\cite{bib:baker,bib:knecht}.

\begin{figure}[h!]
\begin{center}
   \includegraphics[width=0.7\textwidth]{femm_and_calcs.pdf}
    \caption{Magnetic field as a function of relative magnetic
      permeability $\mu_r$ for geometries similar to the ILL nEDM
      experiment.  The dashed line is for an ideal spherical surface
      current of radius 0.53~m inside a spherical shell of inner
      radius 0.57~m, thickness 1.5~mm.  Open and close circles are
      FEMM-based simulations of spherical and cylindrical geometries
      with similar dimensions, described in the text.  The coil
      currents have been arranged to give a $1~\mu$T field at the
      $\mu_r=20,000$ point.}
    \label{fig:Magnetic_Field}
    \end{center}
\end{figure} 


The calculations are compared with finite-element analysis simulations
which were conducted to analyze the effect of discretizing the surface
current, and to test the geometry dependence of the sensitivity of the
experiment to changes in $\mu$.  Two axially symmetric simulations
were conducted using FEMM~\cite{bib:femm}.  In the first simulation,
the same spherical geometry was used as for the analytical
calculations.  However, the surface current was discretized to 50
individual circular wires, inscribed onto a sphere, and equally spaced
vertically (i.e.~a discrete cos$\theta$ coil).  A square wire profile
of side length 1~mm was used.  As shown in
Fig.~\ref{fig:Magnetic_Field}, this simulation gave good agreement
with the analytical calculation.

As an example of one additional axially symmetric geometry, a solenoid
within a cylindrical shield was simulated, with equal coil spacings.
In the limit of infinite $\mu$, the image currents in the end caps of
the shield are an infinite series of coils, giving an ideal infinite
solenoid with a uniform field.  Again, fifty discrete coils were
simulated, where the spacing from an end coil to the inner face of the
shield end-cap being half the inter-coil spacing, as appropriate to
generate the correct image currents in the infinite $\mu$ limit. 
As shown in Fig.~\ref{fig:Magnetic_Field}, the slope of $B_0(\mu)$ is
somewhat steeper, and similar in magnitude to the spherical case.

In ancillary measurements of shielding factors (discussed briefly in
Section~\ref{sec:previousmeasurement}), we found $\mu_r=20,000$ to
offer a reasonable description of the quasistatic shielding factor. By
evaluating the slope of the curve in Fig.~\ref{fig:Magnetic_Field} at
$\mu_r=20,000$, we estimate that the scale of the sensitivity of a
generic nEDM experiment to global changes in the magnetic permeability
is $\frac{\mu}{B_0}\frac{dB_0}{d\mu}\sim 0.01$.

%In the spherical case the reaction factor is 1.39, meaning that the
%field generated by the spherical coil is amplified by this
%multiplicative factor by the presence of the magnetic shield (flux
%return).  In the cylindrical case the reaction factor is 1.35.  If the
%reaction factor is closer to unity, the calculated
%$\frac{\mu}{B_0}\frac{dB_0}{d\mu}$ will be smaller, and the EDM
%experiment less sensitive to changes in $\mu$.  The results in the two
%geometries show that there is also geometry dependence.


For a high-$\mu$ innermost shield, the magnetic field lines emanating
from the coil all return through the shield.  This principle can be
used to estimate the magnetic field internal to the material $B_m$,
and in our studies gave good agreement with FEA-based simulations.
For the cylindrical geometry described in Sec.~\ref{sec:femm}, $B_m$
is largest in the side walls of the solenoidal flux return, attaining
a maximum value of 170~$\mu$T.  If we assume $\mu_r$=20,000, the $H_m$
field is 0.007~A/m.  Typically the shield is degaussed (idealized)
with the internal coil energized.  After degaussing, $B_m$ must be
approximately the same, since all flux returns through the shield.
However, the $H_m$ field must become significantly smaller, as the
material must reside on the ideal magnetization curve in $B_m-H_m$
space.  (For a discussion of the ideal magnetization curve, we refer
the reader to Ref.~\cite{bib:bozorth}.)  In principle, the $H_m$ field
could be reduced by an order of magnitude or more, depending on the
steepness of the ideal magnetization curve near the origin.  Thus
$B_m=170~\mu$T and $H_m<0.007$~A/m set a scale for the relevant values
for nEDM experiments.  Furthermore, the field in the nEDM measurement
volume, as well as in the magnetic shield, must be stable for periods
of typically hundreds of seconds (corresponding to frequencies
$<0.01$~Hz).  This sets the relevant timescale for magnetic properties
most relevant to nEDM experiments.
